\begin{frame}{Haskell}
    \begin{exercise}
        Programmieren Sie folgende Funktionen in Haskell:
        \begin{enumerate}[a)]
            \item inchesToCentimeters
            \item footToCentimeters
            \item circleArea
        \end{enumerate}
    \end{exercise}
\end{frame}
% copy to cleanup my code
\begin{frame}[fragile]{Haskell}
    \begin{solve}[b) inchesToCentimeters]
        \begin{plainhaskell}
!*\onslide<+->*!-- 1 Zoll = 2,54 cm
!*\onslide<+->*!centimetersPerInch :: Float
!*\onslide<.->*!centimetersPerInch = 2.54

!*\onslide<+->*!-- Einfach nur multiplizieren :D
!*\onslide<+->*!inchesToCentimeters :: Float -> Float
!*\onslide<.->*!inchesToCentimeters inches = inches * centimetersPerInch
        \end{plainhaskell}
    \end{solve}
\end{frame}
% copy to cleanup my code
\addtocounter{solve}{-1}% reset counters
\begin{frame}[fragile]{Haskell}
    \begin{solve}[b) footToCentimeters]
        \begin{plainhaskell}
!*\onslide<+->*!-- 1 Fuß = 12 Zoll
!*\onslide<+->*!inchesPerFoot :: Float
!*\onslide<.->*!inchesPerFoot = 12

!*\onslide<+->*!-- Multiplizieren und Umwandeln
!*\onslide<+->*!footToCentimeters :: Float -> Float
!*\onslide<.->*!footToCentimeters foot = inchesToCentimeters (foot * inchesPerFoot)
        \end{plainhaskell}
    \end{solve}
\end{frame}
% copy to cleanup my code
\addtocounter{solve}{-1}% reset counters
\begin{frame}[fragile]{Haskell}
    \begin{solve}[b) circleArea]
        \begin{plainhaskell}
!*\onslide<+->*!-- Radius -> cm -> pi * r^2
!*\onslide<+->*!circleArea :: Float -> Float
!*\onslide<.->*!circleArea radiusFoot = pi * (footToCentimeters radiusFoot)^2
        \end{plainhaskell}
    \end{solve}
\end{frame}