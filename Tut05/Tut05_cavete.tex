\documentclass[aspectratio=169,usepdftitle=true,11pt,ngerman,t]{beamer}

\newcommand\lqfirstname{Name}
\newcommand\lqfullname{Name Last}
\newcommand\lqemail{name.last}

\errorcontextlines999999

\usepackage[T1]{fontenc}
\usepackage[utf8]{inputenc}
\usepackage{microtype}

\usepackage{amsmath,amssymb}
\usepackage{mathtools}
\usepackage{array}
\usepackage{booktabs}
\usepackage{cancel}
\usepackage{ulem}

\usepackage[prefix=]{xcolor-material}
\usepackage{graphics}
\usepackage{tcolorbox}

\usepackage[main=ngerman]{babel}

\usepackage{tikz}
\usetikzlibrary{positioning}
\usetikzlibrary{tikzmark}
\usetikzlibrary{calc}
\usetikzlibrary{automata}

\usepackage{csquotes}
\usepackage{hyperref}
\usepackage{algpseudocode}

% flo
\usepackage{fancyqr}
\usepackage{color-palettes}
\usepackage[encoding,noloadlangs,cpalette,numinpar,fakeminted]{sopra-listings}
\usepackage{code-animation}

\usetheme[theorems]{colorful-dream}

% TODO: cleanup?
\title{Informatik I}
\institute{Universität Münster}

\author{\lqfullname}
\email{\lqemail @uni-muenster.de}

% fix email
\makeatletter
\let\originalhref\href
\renewcommand{\href}[2]{%
    \def\temp{#2}
    \def\tempcompare{\btcd@email}
    \ifx\temp\tempcompare
        \originalhref{mailto:#2}{\hbox{#2}}
    \else
        \originalhref{#1}{#2}
    \fi
}
\makeatother
\addtobeamertemplate{title page}{}{%
\scriptsize\color{gray}\begin{tikzpicture}[overlay, remember picture]
    \node[above left = 1em and 1em of current page.south east,align=center] (n0) {\LaTeX-Vorlage von\\\href{https://github.com/EagleoutIce}{Florian Sihler}};
    \node[above = 0cm of n0] (n1) {\fancyqr[size=1.5cm,tight]{https://eagleoutice.github.io/beamer-themes/\#colorful-dream}};
\end{tikzpicture}}

% TODO: more (specific)?
% better fractions
\newcommand\betterfrac[2]{
    \frac{#1}{\vphantom{#2}#2}
}

% exercise points % TODO
\newcommand\expoints[1]{\null\hfill\texttt{\scriptsize\color{lightgray}(#1)}}
% \omit\expoints{1}\ignorespaces
% a1

% title image: box
% (inverse scale, width, content)
\newcommand\titleimagebox[3]{%
\newlength\tibheight \setlength\tibheight{4cm}
\newlength\tibtY \setlength\tibtY{-6em}
\titleimage{\begin{minipage}[c][#1\tibheight]{#2}\vspace*{#1\tibtY}%
    \begin{tcolorbox}[boxrule=0.1pt]%
    \centering#3% TODO: bug: not centered if too wide
    \end{tcolorbox}%
\end{minipage}}}

% outro image: text
% (text)
\newcommand\outroimagetext[1]{\outro{Münster, \today%
\begin{tikzpicture}[remember picture,overlay]
    \node[align=center] at (current page.center) {#1};
\end{tikzpicture}}
}

% console example calls % TODO
% \newcommand<>\excall[1]{%
% \only<handout:0>{%
% \begin{tikzpicture}[overlay]
%     \node {%
%     \begin{tcolorbox}[boxrule=0.1pt,hbox,boxsep=2pt,left=0pt,right=0pt,top=0pt,bottom=0pt]
%         \ibash[numbers=none,frame=none,backgroundcolor={}]{#1}
%     \end{tcolorbox}
%     };
% \end{tikzpicture}
% }}

% 
% AnimateCode Commands
% 
% X        - zeile (0: all)
% /X:      - kommentar
% .X       - reset
% |X:      - reset + kommentar
%
% +        - continue/walk
% -        - repeat
% 
% oX : {X} - kein marker + weitere zeilen
% :X : {X} - weitere zeilen
% *C       - frei: \Line{X}\Comment{}\Location{start|end|k}\Reset
%
% \StoreAnimationTo\NAnim\StoreHandoutTo\NHandout\StoreTo\N
%
% handout={8/2,\CodeAnimGet{NAnim}/3} - can be used pre definition
%

\solLoadLanguage{haskell}

%%%%%%%%%%%%%%%%%%%%%%%%%%%%%%%%%%%%%%%

\subtitle{Tutorium 5}
\date{15. November 2024}

\titleimagebox{0.5}{4cm}{Aufgepasst!}

\outroimagetext{\Large Interesse an einer \LaTeX-Einführung? :D}

%%%%%%%%%%%%%%%%%%%%%%%%%%%%%%%%%%%%%%%

\begin{document}

\section[Übungsblatt 4]{Übungsblatt 4}

\subsection{Aufgabe 1}
\begin{frame}{Spezifikationen}
    \begin{exercise}[a) `Spezifikation' erfüllt?]
        Lies eine {\only<3>{\bfseries}Zahl} von der Tastatur ein, {\only<5>{\bfseries}berechne die Quadratwurzel}, und {\only<7>{\bfseries}gib das Ergebnis am Bildschirm aus}.
    \end{exercise}
    \pause[]
    \begin{solve}[a)]
    \begin{enumerate}
        \item Vollständig: \onslide<4-8>{Welche Zahlendarstellung? Negative Zahlen? \faTimes}
        \item Detailliert: \onslide<6-8>{Welche Grundoperationen sind erlaubt? \faTimes}
        \item Unzweideutig: \onslide<8>{Was heißt ,,ausgeben''? \faTimes}
    \end{enumerate}
    \end{solve}
\end{frame}
% copy to cleanup my code
\addtocounter{exercise}{-1}\addtocounter{solve}{-1}% reset counters
\begin{frame}{Spezifikationen}
    \begin{exercise}[b) `Spezifikation' erfüllt?]
        Berechne die Länge des Wortes als Binärzahl, und zwar in der Formulierung aus Aufgabe 4 von Übungsblatt 2.
    \end{exercise}
    \begin{solve}[b)]
    \begin{enumerate}
        \item Vollständig: \onslide<3-5>{\faCheck}
        \item Detailliert: \onslide<4-5>{mathematisch formales Maschinenmodell \faCheck}
        \item Unzweideutig: \onslide<5>{Turingmaschine \faCheck}
    \end{enumerate}
    \end{solve}
    \only<2-5>{\begin{tikzpicture}[overlay]
        \node[inner sep=0] (n0) at (current page.east) {};
        \only<2|handout:0>{\node[below left = 3em and 2em of n0] (n1) {\begin{tcolorbox}[boxrule=0.1pt,hbox,boxsep=2pt,left=0pt,right=0pt,top=0pt,bottom=0pt]
            \includegraphics[scale=0.65]{sources/blatt2aufgabe4.png}
        \end{tcolorbox}};}
        \only<3-5>{\node[below left = -1.5em and 3em of n0] (n1) {\begin{tcolorbox}[boxrule=0.1pt,hbox,boxsep=2pt,left=0pt,right=0pt,top=0pt,bottom=0pt]
            \includegraphics[scale=0.2]{sources/blatt2aufgabe4.png}
        \end{tcolorbox}};}
    \end{tikzpicture}}
\end{frame}

\subsection{Aufgabe 2}
\begin{frame}[fragile]{Warum so faul?}
    \begin{exercise}[a)]
    Programmieren Sie eine Funktion, die für einen Integer-Wert prüft, ob dieser eine Primzahl ist.
    \end{exercise}
    \begin{solve}[a)]
    \begin{plainhaskell}
is_prime :: Integer -> Bool
is_prime n
    | n <= 1 = False
    | n == 2 = True
    | n `mod` 2 == 0 = False
    | otherwise = null [ x | x <- [3,5 .. floor (sq!*\tikzmark{sqrt}*!rt (fromIntegral n))], n `mod` x == 0]
    \end{plainhaskell}
    \end{solve}
    \begin{tikzpicture}[overlay,remember picture]
        \node[above=0cm of pic cs:sqrt] (n0) {};
        \node[above=2em of n0,inner sep=2pt,draw,rounded corners,anchor=south] (n1) {\shortstack{$\forall n=x\cdot y:$ \\ $x\geq\sqrt{n}\implies y \leq\sqrt{n}$}};
        \draw (n0) edge[->,dashed] (n1.south);
    \end{tikzpicture}
\end{frame}
\resetframecounters
\begin{frame}[fragile]{Warum so faul?}
    \begin{exercise}[b)]
    Definieren Sie unter Nutzung von \bhaskell{is_prime} eine Liste aller Primzahlen.
    \end{exercise}
    \begin{solve}[b)]
    \begin{plainhaskell}
primes :: [Integer]
primes = [ n | n <- [2..], is_prime n ]
    \end{plainhaskell}
    oder effizienter:
    \begin{plainhaskell}
primes :: [Integer]
primes = [2] ++ [ n | n <- [3,5 ..], is_prime n ]
    \end{plainhaskell}
    \end{solve}
\end{frame}
\resetframecounters
\begin{frame}[fragile]{Warum so faul?}
    \begin{exercise}[c)]
    Extrahieren Sie die 42. Primzahl.
    \end{exercise}
    \begin{solve}[c)]
    \begin{plainhaskell}
prime_no :: Int -> Integer
prime_no n = last (take n primes)
    \end{plainhaskell}
    \begin{plainbash}
> prime_no 42
181
    \end{plainbash}
    \end{solve}
\end{frame}

\subsection{Aufgabe 3}
\begin{frame}[fragile]{Kommandozeile}
    \begin{onlyenv}<1> % do i rly wanna use this?
    \begin{exercise}
        \begin{enumerate}[a)]
            \item Euer Betriebssystem?
            \item Arbeit mit der Kommandozeile
        \end{enumerate}
    \end{exercise}
    \end{onlyenv}
    \pause[]
    \begin{solve}[b) Arbeit mit der Kommandozeile]
        \begin{plainbash}
*\only<.(8)->{\vspace{-\baselineskip}}\onslide<+-.(7)>*mkdir Studium
*\only<.(8)->{\vspace{-\baselineskip}}\onslide<+-.(7)>*cd Studium
*\only<.(7)->{\vspace{-\baselineskip}}\onslide<+-.(7)>*mkdir 2024-WiSe
*\only<.(7)->{\vspace{-\baselineskip}}\onslide<+-.(6)>*mkdir 2024-WiSe/Informatik-I
*\only<.(7)->{\vspace{-\baselineskip}}\onslide<+-.(6)>*cd 2024-WiSe/Informatik-I
*\only<.(7)->{\vspace{-\baselineskip}}\onslide<+-.(6)>*mkdir Folien
*\only<.(7)->{\vspace{-\baselineskip}}\onslide<+-.(6)>*mkdir Übungen
*\only<-.>{\vspace{-\baselineskip}}\onslide<+->*cd ~/Downloads
*\only<-.>{\vspace{-2\baselineskip}}\onslide<+->*mv 00_Organisatorisches_v2.pdf ~/Studium/2024-WiSe/Informatik-I/Folien
*\only<-.>{\vspace{-\baselineskip}}\onslide<+->*cd ~/Studium/2024-WiSe/Informatik-I/Übungen
*\only<-.>{\vspace{-\baselineskip}}\onslide<+->*ghc -o hello hello.hs
*\only<-.>{\vspace{-\baselineskip}}\onslide<+->*ls
*\only<-.>{\vspace{-\baselineskip}}\onslide<+->*/hello
        \end{plainbash}
    \end{solve}
\end{frame}

\subsection{Aufgabe 4}
\only<beamer>{
\begin{frame}[fragile]
    \only<2>{\scriptsize}\only<3->{\Tiny}
    \begin{exercise}[Betrachten Sie das Cavete-Szenario.]
    \begin{itemize}
        \item Weitergehende Studien haben ergeben, dass die Änderungen an den Anzahlen verkaufter Gerichte besser durch ein kubisches Polynom beschrieben werden können als durch die Konstante \T{dec\_portions}, und zwar durch $f(x) = x^3 + ax$ für eine Preiserhöhung um $x$ Euro. (Positive $x$-Werte stellen Preiserhöhungen dar; der Funktionswert gibt dann an, wie viele Portionen weniger verkauft werden. Negative $x$-Werte stellen reduzierte Preise dar, bis hin zu ,,Bezahlungen'' für Bestellungen, und führen zu weiteren Verkäufen.)
        \item Es kommt ein Einkaufsrabatt von $5\%$ auf den \T{price\_per\_portion} zustande, wenn mindestens $150$ Gerichte verkauft werden.
        \item Aufgrund eines beschränkten Lagerraumes für Zutaten kann maximal die Obergrenze von $300$ Gerichten verkauft werden.
    \end{itemize}
    \begin{enumerate}[a)]
        \item erechnen Sie den Wert der Konstante $a$ in obigem Polynom, so dass $f(1) = 10$ (was den Wert $10$ für \T{dec\_portions} verallgemeinert).
        \item Ergänzen Sie das Programm der Vorlesung um geeignete Konstanten für obige Angaben.
        \item Ändern Sie das Programm der Vorlesung, um obige Sachverhalte abzubilden. Folgen Sie dem Vorgehen der Vorlesung, um notwendige weitere Funktionen einzuführen (von Funktionsköpfen über Beispiele, die typische Fälle zeigen und sich leicht berechnen lassen, zu Funktionsrümpfen und exemplarischen Aufrufen). Beachten Sie zudem nachfolgende Vorgaben.
        \begin{itemize}\only<2>{\scriptsize}\only<3->{\Tiny}
            \item Lassen Sie die Signaturen der Funktionen \T{costs} und \T{portions} unverändert. Ändern Sie nur deren Definitionen.
            \item Nutzen Sie \T{if-then-else} in einer der neuen Funktionen, um den Preis pro Portion in Abhängigkeit von der Menge (und damit dem möglichen Rabatt) zu berechnen.
            \item Stellen Sie durch die Verwendung von \textit{Wächtern} in einer der neuen Funktionen sicher, dass weder negative Anzahlen von Gerichten berechnet werden noch mehr als die oben genannte Obergrenze.
            \item Nutzen Sie lediglich in der Vorlesung vorgestellte Haskell-Konstrukte.
        \end{itemize}
    \end{enumerate}
    \end{exercise}
    \begin{solve}
    \ihaskell[numbers=none,frame=none,backgroundcolor={}]{sources/cavete_loesung.hs}
    \end{solve}
    \only<4>{\begin{tikzpicture}[overlay,remember picture]
        \draw[fill=black,fill opacity=0.3] (current page.north west) rectangle (current page.south east);
        \node at (current page.center) {\Huge\twemoji[scale=5]{loudly crying face}};
    \end{tikzpicture}}
\end{frame}
\addtocounter{exercise}{-1}\addtocounter{solve}{-1}% reset counters
}
\begin{frame}[fragile]\onslide<+->% init animation
    \begin{exercise}[a)]
    Programmieren Sie Algorithmus aus Aufgabe 2 von Übungsblatt 3 als rekursive Funktion. Welche Basisfälle lassen sich ableiten?
    \end{exercise}
    \begin{solve}[a)]
    \begin{plainhaskell}
euclid :: Integer -> Integer -> Integer
euclid m n
  | m < 0 || n < 0   = error "Eingabe negativ!"
  | m == 0           = n                -- Basisfall 1
  | n == 0           = m                -- Basisfall 2
  | m > n            = euclid (m - n) n -- rekursive Aufrufe
  | otherwise        = euclid m (n - m)
    \end{plainhaskell}
    \end{solve}
\end{frame}
\resetframecounters
\begin{frame}<handout:0>[fragile,noframenumbering]\onslide<+->% init animation
    \begin{exercise}[a)]
    Programmieren Sie Algorithmus aus Aufgabe 2 von Übungsblatt 3 als rekursive Funktion. Welche Basisfälle lassen sich ableiten?
    \end{exercise}
    \begin{solve}[a) I10\_a6: Näher am Pseudocode geht nicht!]
    \begin{plainhaskell}
euclid :: Integer -> Integer -> Integer
euclid m n
  | m < 0 || n < 0       = error "Eingabe negativ!"
  | m == 0               = n                -- Basisfall 1
  | n /= 0 && m > n      = euclid (m - n) n
  | n /= 0 && otherwise  = euclid m (n - m)
  | otherwise            = m                -- Basisfall 2
    \end{plainhaskell}
    \end{solve}
\end{frame}
\resetframecounters
\begin{frame}[noframenumbering]\onslide<+->% init animation
    \begin{exercise}[b)]
    Wieso terminiert Ihr Algorithmus?
    \end{exercise}
    \begin{solve}[b)]
    Im Fehlerfall oder bei Erreichen eines Basisfalls endet der Algorithmus.\par\bigskip
    In den rekursiven Aufrufen wird einer der Parameter um eine Zahl ungleich $0$ reduziert. Dies kann nur endlich oft geschehen, bevor ein Basisfall mit Terminierung erreicht wird.
    \onslide<+-|handout:0>\par\bigskip
    \textit{\lqfirstname s Ergänzung: (inpiriert von I10\_a4)}\par
    Es wird immer um eine Zahl reduziert, die kleiner als der Minuend ist. Damit kann der Fehlerfall nur beim initialen Aufruf erreicht werden; ansonsten wird immer schließlich ein Basisfall erreicht!
    \end{solve}
\end{frame}

\section{Anhang}

\subsection{\LaTeX~Übungsblatt Vorlage}
\begin{frame}{Anhang}
    \begin{itemize}
        \item Es gibt eine \LaTeX-Vorlage für Übungsblätter von der AG Vahrenhold:\\\medskip\centering\fancyqr{https://zivgitlab.uni-muenster.de/ag-vahrenhold/public/latex-templates/-/tree/master/abgaben-minted}\makebox[0pt][l]{ (\href{https://zivgitlab.uni-muenster.de/ag-vahrenhold/public/latex-templates/-/tree/master/abgaben-minted}{hier} die \texttt{minted}-Version)}
    \end{itemize}
\end{frame}

\end{document}