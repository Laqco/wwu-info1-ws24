\documentclass[aspectratio=169,usepdftitle=true,11pt,ngerman,t]{beamer}

\newcommand\lqfirstname{Name}
\newcommand\lqfullname{Name Last}
\newcommand\lqemail{name.last}

\errorcontextlines999999

\usepackage[T1]{fontenc}
\usepackage[utf8]{inputenc}
\usepackage{microtype}

\usepackage{amsmath,amssymb}
\usepackage{mathtools}
\usepackage{array}
\usepackage{booktabs}
\usepackage{cancel}
\usepackage{ulem}

\usepackage[prefix=]{xcolor-material}
\usepackage{graphics}
\usepackage{tcolorbox}

\usepackage[main=ngerman]{babel}

\usepackage{tikz}
\usetikzlibrary{positioning}
\usetikzlibrary{tikzmark}
\usetikzlibrary{calc}
\usetikzlibrary{automata}

\usepackage{csquotes}
\usepackage{hyperref}
\usepackage{algpseudocode}

% flo
\usepackage{fancyqr}
\usepackage{color-palettes}
\usepackage[encoding,noloadlangs,cpalette,numinpar,fakeminted]{sopra-listings}
\usepackage{code-animation}

\usetheme[theorems]{colorful-dream}

% TODO: cleanup?
\title{Informatik I}
\institute{Universität Münster}

\author{\lqfullname}
\email{\lqemail @uni-muenster.de}

\addtobeamertemplate{title page}{}{%
\scriptsize\color{gray}\begin{tikzpicture}[overlay, remember picture]
    \node[above left = 1em and 1em of current page.south east,align=center] (n0) {\LaTeX-Vorlage von\\\href{https://github.com/EagleoutIce}{Florian Sihler}};
    \node[above = 0cm of n0] (n1) {\fancyqr[size=1.5cm,tight]{https://eagleoutice.github.io/beamer-themes/\#colorful-dream}};
\end{tikzpicture}}

% TODO: more (specific)?
% better fractions
\newcommand\betterfrac[2]{
    \frac{#1}{\vphantom{#2}#2}
}

% exercise points % TODO
\newcommand\expoints[1]{\null\hfill\texttt{\scriptsize\color{lightgray}(#1)}}
% \omit\expoints{1}\ignorespaces
% a1

% title image: box
% ([vshift], inverse scale, width, content)
\newcommand\titleimagebox[4][0pt]{%
\newlength\tibheight \setlength\tibheight{4cm}
\newlength\tibtY \setlength\tibtY{-6em}
\titleimage{\begin{minipage}[c][#2\tibheight]{#3}\vspace*{#2\tibtY}%
    \vspace*{#1}\begin{tcolorbox}[boxrule=0.1pt]%
    \centering#4% TODO: bug: not centered if too wide
    \end{tcolorbox}%
\end{minipage}}}

% outro image: text
% (text)
\newcommand\outroimagetext[1]{\outro{Münster, \today%
\begin{tikzpicture}[remember picture,overlay]
    \node[align=center] at (current page.center) {#1};
\end{tikzpicture}}
}

% reset counters
\newcommand\resetframecounters{\addtocounter{exercise}{-1}\addtocounter{solve}{-1}}
\newcommand\resetsolve{\addtocounter{solve}{-1}}

% listing style for file input listings
\lstdefinestyle{files}{numbersep=-8pt,xleftmargin=-8pt,frame=none,numbers=none,backgroundcolor={}}

% small footnote
\newcommand{\nicefootnote}[1]{\tikz[overlay,remember picture]{\node[anchor=south,text width=\dimexpr\textwidth+2em\relax,align=center,above=1em of current page.south] (nicefootnote) {\hrule\vskip 1pt\footnotesize#1};}}

% italic for fonts that don't support itshape
\newcommand{\itforce}[1]{\tikz[baseline]{\node[xslant=0.2,anchor=base,inner sep=0pt]{#1};}}

% console example calls % TODO
% \newcommand<>\excall[1]{%
% \only<handout:0>{%
% \begin{tikzpicture}[overlay]
%     \node {%
%     \begin{tcolorbox}[boxrule=0.1pt,hbox,boxsep=2pt,left=0pt,right=0pt,top=0pt,bottom=0pt]
%         \ibash[numbers=none,frame=none,backgroundcolor={}]{#1}
%     \end{tcolorbox}
%     };
% \end{tikzpicture}
% }}

% 
% AnimateCode Commands
% 
% X        - zeile (0: all)
% /X:      - kommentar
% .X       - reset
% |X:      - reset + kommentar
%
% +        - continue/walk
% -        - repeat
% 
% oX : {X} - kein marker + weitere zeilen
% :X : {X} - weitere zeilen
% *C       - frei: \Line{X}\Comment{}\Location{start|end|k}\Reset
%
% \StoreAnimationTo\NAnim\StoreHandoutTo\NHandout\StoreTo\N
%
% handout={8/2,\CodeAnimGet{NAnim}/3} - can be used pre definition
%

\solLoadLanguage{haskell}

%%%%%%%%%%%%%%%%%%%%%%%%%%%%%%%%%%%%%%%

\subtitle{Tutorium 5}
\date{15. November 2024}

\titleimagebox{0.5}{4cm}{Aufgepasst!}

\outroimagetext{\Large Interesse an einer \LaTeX-Einführung? :D}

%%%%%%%%%%%%%%%%%%%%%%%%%%%%%%%%%%%%%%%

\begin{document}

\section[Übungsblatt 4]{Übungsblatt 4}

\subsection{Aufgabe 1}
\begin{frame}
    \begin{exercise}[\textit{schrittweise} Klammerung]
        Setzen Sie Klammern, um zu zeigen, in welcher Reihenfolge die Teilausdrücke ausgewertet werden, und geben Sie das Ergebnis der Auswertung an.
    \end{exercise}
    \begin{solve}[a) \lstcolorlet{numbers}{paletteD}\color{lightgray}\bhaskell{3 * 4 - 2 ^ 2 ^ 4}]
    \begin{itemize}
        \item \textbf{Potenz}: Bindungsstärke $8$, rechts-assoziativ:\hfill\bhaskell{3 * 4 - (2 ^ (2 ^ 4))}
        \item \textbf{Multiplikation}: Bindungsstärke $7$:\hfill\bhaskell{(3 * 4) - (2 ^ (2 ^ 4))}
        \item \textbf{Subtraktion}: Bindungsstärke $6$:\hfill\bhaskell{(3 * 4) - (2 ^ (2 ^ 4))}\\
        \hfill$12 - 65536 = -65524$
    \end{itemize}
    \end{solve}
\end{frame}

\subsection{Aufgabe 2}
\begin{frame}[fragile]
    \begin{exercise}[a) + b)]
        Programmieren Sie in Haskell einen zweistelligen, links-assoziativen Infix-Operator \textasciitilde\textasciitilde~mit der Bindungsstärke 5, der den Mittelwert zweier Float-Zahlen berechnet.
    \end{exercise}
    \begin{solve}[a) + b)]
    \begin{plainhaskell}
-- Bindungsstärke 5
infixl 5 ~~

-- Operator im Klammern
(~~) :: Float -> Float -> Float

-- Mittelwert: Summe, geteilt durch 2
x ~~ y = (x + y)/2
    \end{plainhaskell}
    \end{solve}
\end{frame}
% copy to cleanup my code
\addtocounter{exercise}{-1}\addtocounter{solve}{-1}% reset counters
\begin{frame}
    \begin{exercise}[c)]
        Geben Sie an, wie der Ausdruck \bhaskell{3 + 4 ~~ 5 * 6} ausgewertet wird. Welche Änderungen würden sich bei den Bindungsstärken $6$ oder $7$ ergeben?
    \end{exercise}
    \begin{solve}[c)]
    \begin{itemize}
        \item Bindungsstärke $5$: \bhaskell{*} \faAngleRight~\bhaskell{+} \faAngleRight{}~\bhaskell{\~~}:\hfill\bhaskell{(3 + 4) ~~ (5 * 6)}$= 18.5$
        \item Bindungsstärke $6$: \bhaskell{*} \faAngleRight~\bhaskell{+}$\mid$\bhaskell{\~~}, also links-assoziativ:\\\hfill\bhaskell{(3 + 4) ~~ (5 * 6)}$=18.5$
        \item Bindungsstärke $7$: \bhaskell{*}$\mid$\bhaskell{\~~} \faAngleRight~\bhaskell{+}, also links-assoziativ:\\\hfill\bhaskell{3 + ((4 ~~ 5) * 6)}$=30$
    \end{itemize}
    \end{solve}
\end{frame}

\subsection{Aufgabe 3}
\begin{frame}[fragile]\onslide<+->% init animation
    \begin{exercise}
    Programmieren Sie zweistellige Boole'sche Funktion \T{quiet\_atmosphere}, die zurückgeben soll, ob es ruhig genug ist, um der Vorlesung zu folgen.
    \end{exercise}
    \begin{solve}[mit \T{UND} und \T{NICHT}]
    \begin{plainhaskell}
quiet_atmosphere :: Bool -> Bool -> Bool
quiet_atmosphere quiet_neighbors construction_noise = quiet_neighbors && not construction_noise
    \end{plainhaskell}
    \end{solve}
\end{frame}
\resetframecounters
\begin{frame}[noframenumbering,fragile]\onslide<+->% init animation
    \begin{exercise}
    Programmieren Sie zweistellige Boole'sche Funktion \T{quiet\_atmosphere}, die zurückgeben soll, ob es ruhig genug ist, um der Vorlesung zu folgen.
    \end{exercise}
    \begin{solve}[mit \T{ODER} und \T{NICHT}]
    \begin{plainhaskell}
quiet_atmosphere :: Bool -> Bool -> Bool
quiet_atmosphere quiet_neighbors construction_noise = !*\sol@styles@lst@comments\sol@lst@prebreak*!
   not (not quiet_neighbors || construction_noise)
    \end{plainhaskell}
    \end{solve}
\end{frame}

\subsection{Aufgabe 4}
\only<beamer>{
\begin{frame}[fragile]
    \only<2>{\scriptsize}\only<3->{\Tiny}
    \begin{exercise}[Betrachten Sie das Cavete-Szenario.]
    \begin{itemize}
        \item Weitergehende Studien haben ergeben, dass die Änderungen an den Anzahlen verkaufter Gerichte besser durch ein kubisches Polynom beschrieben werden können als durch die Konstante \T{dec\_portions}, und zwar durch $f(x) = x^3 + ax$ für eine Preiserhöhung um $x$ Euro. (Positive $x$-Werte stellen Preiserhöhungen dar; der Funktionswert gibt dann an, wie viele Portionen weniger verkauft werden. Negative $x$-Werte stellen reduzierte Preise dar, bis hin zu ,,Bezahlungen'' für Bestellungen, und führen zu weiteren Verkäufen.)
        \item Es kommt ein Einkaufsrabatt von $5\%$ auf den \T{price\_per\_portion} zustande, wenn mindestens $150$ Gerichte verkauft werden.
        \item Aufgrund eines beschränkten Lagerraumes für Zutaten kann maximal die Obergrenze von $300$ Gerichten verkauft werden.
    \end{itemize}
    \begin{enumerate}[a)]
        \item erechnen Sie den Wert der Konstante $a$ in obigem Polynom, so dass $f(1) = 10$ (was den Wert $10$ für \T{dec\_portions} verallgemeinert).
        \item Ergänzen Sie das Programm der Vorlesung um geeignete Konstanten für obige Angaben.
        \item Ändern Sie das Programm der Vorlesung, um obige Sachverhalte abzubilden. Folgen Sie dem Vorgehen der Vorlesung, um notwendige weitere Funktionen einzuführen (von Funktionsköpfen über Beispiele, die typische Fälle zeigen und sich leicht berechnen lassen, zu Funktionsrümpfen und exemplarischen Aufrufen). Beachten Sie zudem nachfolgende Vorgaben.
        \begin{itemize}\only<2>{\scriptsize}\only<3->{\Tiny}
            \item Lassen Sie die Signaturen der Funktionen \T{costs} und \T{portions} unverändert. Ändern Sie nur deren Definitionen.
            \item Nutzen Sie \T{if-then-else} in einer der neuen Funktionen, um den Preis pro Portion in Abhängigkeit von der Menge (und damit dem möglichen Rabatt) zu berechnen.
            \item Stellen Sie durch die Verwendung von \textit{Wächtern} in einer der neuen Funktionen sicher, dass weder negative Anzahlen von Gerichten berechnet werden noch mehr als die oben genannte Obergrenze.
            \item Nutzen Sie lediglich in der Vorlesung vorgestellte Haskell-Konstrukte.
        \end{itemize}
    \end{enumerate}
    \end{exercise}
    \begin{solve}
    \ihaskell[numbers=none,frame=none,backgroundcolor={}]{sources/cavete_loesung.hs}
    \end{solve}
    \only<4>{\begin{tikzpicture}[overlay,remember picture]
        \draw[fill=black,fill opacity=0.3] (current page.north west) rectangle (current page.south east);
        \node at (current page.center) {\Huge\twemoji[scale=5]{loudly crying face}};
    \end{tikzpicture}}
\end{frame}
\addtocounter{exercise}{-1}\addtocounter{solve}{-1}% reset counters
}
\begin{frame}[fragile]{Jonglieren mit Variablen}
    \begin{exercise}
    Nutzen Sie in dieser Aufgabe die Kommandozeile zum Kompilieren und Ausführen von Java-Code.
    \medskip
    Lesen Sie zwei Zahlen von der Kommandozeile ein und speichern Sie diese als Variablen. Tauschen Sie die Werte der beiden Variablen und geben Sie den größeren Wert aus.
    \end{exercise}
\end{frame}
\begin{frame}[fragile]{Jonglieren mit Variablen}
    \begin{solve}[a) + b) + c)]
    \begin{plainjava}
// Deklaration (a)
int x, y;

// Eingabe (b)
x = IOTools.readInt("Eingabe x: ");
y = IOTools.readInt("Eingabe y: ");

// Ausgabe (c)
System.out.println("x: " + x + " - y: " + y);
    \end{plainjava}
    \end{solve}
\end{frame}
\resetframecounters
\begin{frame}[fragile]{Jonglieren mit Variablen}
    \begin{solve}[d)]
    \begin{plainjava}
// Tauschen (mit Hilfsvariable)
int z = x;
x = y;
y = z;

// Ausgabe
System.out.println("x: " + x + " - y: " + y);
    \end{plainjava}
    \end{solve}
\end{frame}
\resetframecounters
\begin{frame}[fragile]{Jonglieren mit Variablen}
    \begin{solve}[e)]
    \begin{plainjava}
// Vergleich
int largerValue;
if (x >= y)
    largerValue = x;
else
    largerValue = y;

System.out.println("Der größere Wert von beiden ist: " + largerValue);
    \end{plainjava}
\end{solve}
Ja, das ist eigentlich nur der ,,nicht-kleinere'' Wert!
\end{frame}
\resetframecounters
\begin{frame}[fragile]{Jonglieren mit Variablen}
    \vspace{-0.5\baselineskip}
    \begin{solve}[d) + e), aber schöner!]
    \begin{plainjava}
// Tauschen (mit XOR)
x = x ^ y;
y = x ^ y; // x ^ y ^ y = x
x = x ^ y; // x ^ y ^ x = y

// besserer Vergleich
int largerValue = x;
if (y > x) largerValue = y;

// oder direkt mit ternärem Operator
System.out.println("Der größere Wert von beiden ist: " + (x >= y) ? x : y);
    \end{plainjava}
    \end{solve}
\end{frame}

\section{Anhang}

\subsection{\LaTeX~Übungsblatt Vorlage}
\begin{frame}{Anhang}
    \begin{itemize}
        \item Es gibt eine \LaTeX-Vorlage für Übungsblätter von der AG Vahrenhold:\\\medskip\centering\fancyqr{https://zivgitlab.uni-muenster.de/ag-vahrenhold/public/latex-templates/-/tree/master/abgaben-minted}\makebox[0pt][l]{ (\href{https://zivgitlab.uni-muenster.de/ag-vahrenhold/public/latex-templates/-/tree/master/abgaben-minted}{hier} die \texttt{minted}-Version)}
    \end{itemize}
\end{frame}

\end{document}