\begin{frame}[fragile]\onslide<+->% init animation
    \begin{exercise}[a) + b)]
        Programmieren Sie in Haskell einen zweistelligen, links-assoziativen Infix-Operator \textasciitilde\textasciitilde~mit der Bindungsstärke 5, der den Mittelwert zweier Float-Zahlen berechnet.
    \end{exercise}
    \begin{solve}[a) + b)]
    \begin{plainhaskell}
!*\onslide<+->*!-- Bindungsstärke 5
infixl 5 ~~

!*\onslide<+->*!-- Signatur, Operator in Klammern
(~~) :: Float -> Float -> Float

!*\onslide<+->*!-- Mittelwert: Summe, geteilt durch 2
x ~~ y = (x + y)/2
    \end{plainhaskell}
    \end{solve}
\end{frame}
% copy to cleanup my code
\addtocounter{exercise}{-1}\addtocounter{solve}{-1}% reset counters
\begin{frame}\onslide<+->% init animation
    \begin{exercise}[c)]
        Geben Sie an, wie der Ausdruck \bhaskell{3 + 4 ~~ 5 * 6} ausgewertet wird. Welche Änderungen würden sich bei den Bindungsstärken $6$ oder $7$ ergeben?
    \end{exercise}
    \begin{solve}[c)]
    \begin{itemize}
        \item<+-> Bindungsstärke $5$: \bhaskell{*} \faAngleRight~\bhaskell{+} \faAngleRight{}~\bhaskell{\~~}:\hfill\bhaskell{(3 + 4) ~~ (5 * 6)}$= 18.5$
        \item<+-> Bindungsstärke $6$: \bhaskell{*} \faAngleRight~\bhaskell{+}$\mid$\bhaskell{\~~}, also links-assoziativ:\\\hfill\bhaskell{(3 + 4) ~~ (5 * 6)}$=18.5$
        \item<+-> Bindungsstärke $7$: \bhaskell{*}$\mid$\bhaskell{\~~} \faAngleRight~\bhaskell{+}, also links-assoziativ:\\\hfill\bhaskell{3 + ((4 ~~ 5) * 6)}$=30$
    \end{itemize}
    \end{solve}
\end{frame}