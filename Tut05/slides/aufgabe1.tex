\begin{frame}
    \begin{exercise}[\textit{schrittweise} Klammerung]
        Setzen Sie Klammern, um zu zeigen, in welcher Reihenfolge die Teilausdrücke ausgewertet werden, und geben Sie das Ergebnis der Auswertung an.
    \end{exercise}
    \begin{solve}[a) \lstcolorlet{numbers}{paletteD}\color{lightgray}\bhaskell{3 * 4 - 2 ^ 2 ^ 4}]
    \begin{itemize}
        \item \textbf{Potenz}: Bindungsstärke $8$, rechts-assoziativ\hfill\bhaskell{3 * 4 - (2 ^ (2 ^ 4))}
        \item \textbf{Multiplikation}: Bindungsstärke $7$\hfill\bhaskell{(3 * 4) - (2 ^ (2 ^ 4))}
        \item \textbf{Subtraktion}: Bindungsstärke $6$\hfill\bhaskell{(3 * 4) - (2 ^ (2 ^ 4))}\\
        \hfill$= -65524$
    \end{itemize}
    \end{solve}
\end{frame}
% copy to cleanup my code
\addtocounter{exercise}{-1}\addtocounter{solve}{-1}% reset counters
\begin{frame}
    \begin{exercise}[\textit{schrittweise} Klammerung]
        Setzen Sie Klammern, um zu zeigen, in welcher Reihenfolge die Teilausdrücke ausgewertet werden, und geben Sie das Ergebnis der Auswertung an.
    \end{exercise}
    \begin{solve}[b) \lstcolorlet{numbers}{paletteD}\color{lightgray}\bhaskell{add 5 3 + add 7 4 * 2 - 3}]
    \begin{itemize}
        \item \textbf{Funktion}: Bindungsstärke $10$\hfill\bhaskell{(add 5 3) + (add 7 4) * 2 - 3}
        \item \textbf{Multiplikation}: Bindungsstärke $7$\hfill\bhaskell{(add 5 3) + ((add 7 4) * 2) - 3}
        \item \textbf{Addition/Subtraktion}: Bindungsstärke $6$, daher Assoziativität (hier: links)\\\hfill\bhaskell{((add 5 3) + ((add 7 4) * 2)) - 3}\\
        \hfill$= 27$
    \end{itemize}
    \end{solve}
\end{frame}