\begin{frame}\onslide<+->% init animation
    \begin{exercise}[a)]
        Die Veränderung der verkauften Portionen bei einer Preiserhöhung um $x$ wird besser durch die Funktion $f(x) = x^3 + ax$ beschrieben.
    \end{exercise}
    \begin{solve}[a) Berechnen Sie $a$, so dass $f(1)=10$.]\vspace{-\abovedisplayskip}\onslide<+->
        $$f(x)=x^3+ax$$
        $$f(1)=1^3+a\cdot1$$
        $$1+a=10 \implies a=9$$
    \end{solve}
\end{frame}
% copy to cleanup my code
\addtocounter{exercise}{-1}\addtocounter{solve}{-1}% reset counters
\begin{frame}[fragile]\onslide<+->% init animation
    \begin{exercise}[b)]
        Es können maximal $300$ Gerichte verkauft werden, mit einem Mengenrabatt von $5\%$ ab $150$ Gerichten.
    \end{exercise}
    \begin{solve}[b) Ergänzen Sie das Programm um geeignete Konstanten]
    \vspace{-\abovedisplayskip}\begin{minipage}[t]{0.5\linewidth}
    \begin{plainhaskell}
!*\onslide<+->*!-- Mengenrabatt
discount :: Double
discount = 0.05

!*\onslide<+->*!-- Schwellwert
discount_threshold :: Int
discount_threshold = 150
    \end{plainhaskell}
    \end{minipage}\begin{minipage}[t]{0.5\linewidth}
    \begin{plainhaskell}
!*\onslide<+->*!-- Maximalmenge
max_portions :: Int
max_portions = 300
    \end{plainhaskell}
    \end{minipage}
    \end{solve}
\end{frame}
% copy to cleanup my code
\addtocounter{exercise}{-1}\addtocounter{solve}{-1}% reset counters
\begin{frame}[fragile]\onslide<+->% init animation
    \begin{exercise}[c)]
        Passen Sie das Programm an und führen Sie neue Funktionen mit passenden Beispielen ein.
    \end{exercise}\onslide<+->
    \begin{solve}[c) Wir haben bereits...]
    \vspace{-\abovedisplayskip}\begin{minipage}[t]{0.5\linewidth}
    \begin{plainhaskell}
-- Mengenrabatt
discount :: Double
discount = 0.05

-- Schwellwert
discount_threshold :: Int
discount_threshold = 150
    \end{plainhaskell}
    \end{minipage}\begin{minipage}[t]{0.5\linewidth}
    \begin{plainhaskell}
-- Maximalmenge
max_portions :: Int
max_portions = 300
    \end{plainhaskell}
    \end{minipage}
    \end{solve}
\end{frame}
% copy to cleanup my code
\addtocounter{exercise}{-1}\addtocounter{solve}{-1}% reset counters
\begin{frame}[noframenumbering,fragile]\onslide<+->% init animation
    \begin{exercise}[c)]
        Passen Sie das Programm an und führen Sie neue Funktionen mit passenden Beispielen ein.
    \end{exercise}
    \begin{solve}[c) Einfach übernehmen:]
    \vspace{-\abovedisplayskip}\begin{minipage}[t]{0.5\linewidth}
    \begin{plainhaskell}
-- Einkaufspreis
price_per_portion :: Double
price_per_portion = 2.0

-- Grundkosten
base_costs :: Double
base_costs = 200.0
    \end{plainhaskell}
    \end{minipage}\begin{minipage}[t]{0.5\linewidth}
    \begin{plainhaskell}
-- aktueller Verkaufspreis
base_price :: Double
base_price = 9.90

-- Grundanzahl Portionen
base_portions :: Int
base_portions = 100
    \end{plainhaskell}
    \end{minipage}
    \end{solve}
\end{frame}
% copy to cleanup my code
\addtocounter{exercise}{-1}\addtocounter{solve}{-1}% reset counters
\begin{frame}[noframenumbering,fragile]\onslide<+->% init animation
    \begin{exercise}[c)]
        Passen Sie das Programm an und führen Sie neue Funktionen mit passenden Beispielen ein.
    \end{exercise}
    \begin{solve}[c) Die neue Formel:]
    \begin{plainhaskell}
!*\onslide<+->*!-- Verkaufsrückgang pro Euro Preisanstieg
demand_decrease :: Double -> Int
demand_decrease delta = round (delta^3 + 9*delta)
!*\onslide<+->*!
{-
-- wir entfernen dafür dec_portions
dec_portions :: Int
dec_portions = 10
-}
    \end{plainhaskell}
    \end{solve}
\end{frame}
% copy to cleanup my code
\addtocounter{exercise}{-1}\addtocounter{solve}{-1}% reset counters
\begin{frame}[noframenumbering,fragile]\onslide<+->% init animation
    \begin{exercise}[c)]
        Passen Sie das Programm an und führen Sie neue Funktionen mit passenden Beispielen ein.
    \end{exercise}
    \begin{solve}[c) Obergrenze für den Verkauf:]
    \begin{plainhaskell}
!*\onslide<+->*!-- Beschränkt die Anzahl der Portionen
capped_portions :: Int -> Int
capped_portions portions
    | portions < 0            = 0
    | portions > max_portions = max_portions
    | otherwise               = portions
    \end{plainhaskell}
    \end{solve}
\end{frame}
% copy to cleanup my code
\addtocounter{exercise}{-1}\addtocounter{solve}{-1}% reset counters
\begin{frame}[noframenumbering,fragile]\onslide<+->% init animation
    \begin{exercise}[c)]
        Passen Sie das Programm an und führen Sie neue Funktionen mit passenden Beispielen ein.
    \end{exercise}
    \begin{solve}[c) Rabatt einberechnen:]
    \begin{plainhaskell}
!*\onslide<+->*!-- Berechnet Preis inklusive Rabatt
portion_price :: Int -> Double
portion_price meals =
    if meals >= discount_threshold
    then price_per_portion * (1 - discount)
    else price_per_portion
    \end{plainhaskell}
    \end{solve}
\end{frame}
% copy to cleanup my code
\addtocounter{exercise}{-1}\addtocounter{solve}{-1}% reset counters
\begin{frame}[noframenumbering,fragile]\onslide<+->% init animation
    \begin{exercise}[c)]
        Passen Sie das Programm an und führen Sie neue Funktionen mit passenden Beispielen ein.
    \end{exercise}
    \begin{solve}[c) \T{portions} und \T{costs}:]
    \begin{plainhaskell}
!*\onslide<+->*!-- Verkaufte Portionen für einen Preis
portions :: Double -> Int
portions price = capped_portions (base_portions - demand_decrease (price - base_price))

!*\onslide<+->*!-- Ausgaben für einen Preis
costs :: Double -> Double
costs price = base_costs + fromIntegral (portions price) * portion_price (portions price)
    \end{plainhaskell}
    \end{solve}
\end{frame}
% copy to cleanup my code
\addtocounter{exercise}{-1}\addtocounter{solve}{-1}% reset counters
\begin{frame}[noframenumbering,fragile]\onslide<+->% init animation
    \begin{exercise}[c)]
        Passen Sie das Programm an und führen Sie neue Funktionen mit passenden Beispielen ein.
    \end{exercise}
    \begin{solve}[c) Der Rest:]
    \begin{plainhaskell}
-- Einnahmen für einen Preis 
revenue :: Double -> Double
revenue price = price * fromIntegral (portions price)

-- Gewinn für einen Preis
profit :: Double -> Double
profit price = (revenue price) - (costs price)
    \end{plainhaskell}
    \end{solve}
\end{frame}