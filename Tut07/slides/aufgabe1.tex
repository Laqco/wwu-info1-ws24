\begin{frame}[fragile]{Mengen in Haskell}
    Wir möchten Mengen\tikzmark{mengen} in Haskell verwalten.

    \begin{tikzpicture}[overlay,remember picture]
        \node[below = 1em of pic cs:mengen,inner sep=2pt,draw,rounded corners,anchor=west] (n0) {\textbf{Keine Reihenfolge und keine Duplikate!}};
        \node[below left = 2px and 2em of pic cs:mengen,inner sep=0pt] (n1) {};
        \draw[thick] (n1) edge[->,bend right=30] (n0.west);
    \end{tikzpicture}
    
    \begin{exercise}[a)]
    \bhaskell{is_empty}: Prüft, ob eine Menge leer ist.
    \end{exercise}
    \begin{solve}[a)]
    \begin{plainhaskell}
is_empty :: [Integer] -> Bool
is_empty [] = True
is_empty _ = False
    \end{plainhaskell}
    \end{solve}
\end{frame}
\resetframecounters
\begin{frame}[fragile]{Mengen in Haskell}
    \begin{exercise}[b)]
        \bhaskell{is_elem}: Prüft, ob ein Wert in einer Menge enthalten ist.
        \end{exercise}
        \begin{solve}[b)]
        \begin{plainhaskell}
is_elem :: Integer -> [Integer] -> Bool
is_elem _ [] = False
is_elem x (y:ys)
    | x == y = True
    | otherwise = is_elem x ys
        \end{plainhaskell}
        \end{solve}
\end{frame}
\resetframecounters
\begin{frame}[fragile]{Mengen in Haskell}
    \begin{exercise}[c)]
        \bhaskell{is_subset}: Prüft, ob die erste Menge in der zweiten enthalten ist.
        \end{exercise}
        \begin{solve}[c)]
        \begin{plainhaskell}
is_subset :: [Integer] -> [Integer] -> Bool
is_subset [] _ = True
is_subset _ [] = False
is_subset (x:xs) set2
    | is_elem x set2 = is_subset xs set2
    | otherwise = False
        \end{plainhaskell}
        \end{solve}
\end{frame}
\resetframecounters
\begin{frame}[fragile]{Mengen in Haskell}
    \begin{exercise}[d)]
        \bhaskell{is_equal}: Prüft, ob zwei Mengen gleich sind.
        \end{exercise}
        \begin{solve}[d)]
        \begin{plainhaskell}
is_equal :: [Integer] -> [Integer] -> Bool
is_equal set1 set2 = is_subset set1 set2 && is_subset set2 set1
        \end{plainhaskell}
        \end{solve}
\end{frame}
\resetframecounters
\begin{frame}[fragile]{Mengen in Haskell}
    \begin{exercise}[e)]
        \bhaskell{delete}: Löscht ein Element aus einer Menge.
        \end{exercise}
        \begin{solve}[e)]
        \begin{plainhaskell}
delete :: Integer -> [Integer] -> [Integer]
delete _ [] = []
delete x (y:ys)
    | x == y = ys
    | otherwise = y : delete x ys
        \end{plainhaskell}
        \end{solve}
\end{frame}
\resetframecounters
\begin{frame}[fragile]{Mengen in Haskell}
    \begin{exercise}[f)]
        \bhaskell{insert}: Fügt ein Element zu einer Menge hinzu.
        \end{exercise}
        \begin{solve}[f)]
        \begin{plainhaskell}
insert :: Integer -> [Integer] -> [Integer]
insert x [] = [x]
insert x set
    | is_elem x set = set
    | otherwise = x : set
        \end{plainhaskell}
        \end{solve}
\end{frame}
\resetframecounters
\begin{frame}[fragile]{Mengen in Haskell}
    \begin{exercise}[g)]
        \bhaskell{union}: Bildet die Vereinigung zweier Mengen.
        \end{exercise}
        \begin{solve}[g)]
        \begin{plainhaskell}
union :: [Integer] -> [Integer] -> [Integer]
union [] set2 = set2
union (x:xs) set2 = insert x (union xs set2)
        \end{plainhaskell}
        \end{solve}
\end{frame}