\begin{frame}[fragile]{Jonglieren mit Variablen}
    \begin{exercise}
    Nutzen Sie in dieser Aufgabe die Kommandozeile zum Kompilieren und Ausführen von Java-Code.
    \medskip
    Lesen Sie zwei Zahlen von der Kommandozeile ein und speichern Sie diese als Variablen. Tauschen Sie die Werte der beiden Variablen und geben Sie den größeren Wert aus.
    \end{exercise}
\end{frame}
\begin{frame}[fragile]{Jonglieren mit Variablen}
    \begin{solve}[a) + b) + c)]
    \begin{plainjava}
// Deklaration (a)
int x, y;

// Eingabe (b)
x = IOTools.readInt("Eingabe x: ");
y = IOTools.readInt("Eingabe y: ");

// Ausgabe (c)
System.out.println("x: " + x + " - y: " + y);
    \end{plainjava}
    \end{solve}
\end{frame}
\resetframecounters
\begin{frame}[fragile]{Jonglieren mit Variablen}
    \begin{solve}[d)]
    \begin{plainjava}
// Tauschen (mit Hilfsvariable)
int z = x;
x = y;
y = z;

// Ausgabe
System.out.println("x: " + x + " - y: " + y);
    \end{plainjava}
    \end{solve}
\end{frame}
\resetframecounters
\begin{frame}[fragile]{Jonglieren mit Variablen}
    \begin{solve}[e)]
    \begin{plainjava}
// Vergleich
int largerValue;
if (x >= y)
    largerValue = x;
else
    largerValue = y;

System.out.println("Der größere Wert von beiden ist: " + largerValue);
    \end{plainjava}
\end{solve}
Ja, das ist eigentlich nur der ,,nicht-kleinere'' Wert!
\end{frame}
\resetframecounters
\begin{frame}[fragile]{Jonglieren mit Variablen}
    \vspace{-0.5\baselineskip}
    \begin{solve}[d) + e), aber schöner!]
    \begin{plainjava}
// Tauschen (mit XOR)
x = x ^ y;
y = x ^ y; // x ^ y ^ y = x
x = x ^ y; // x ^ y ^ x = y

// besserer Vergleich
int largerValue = x;
if (y > x) largerValue = y;

// oder direkt mit ternärem Operator
System.out.println("Der größere Wert von beiden ist: " + (x >= y) ? x : y);
    \end{plainjava}
    \end{solve}
\end{frame}