\begin{frame}[fragile]{Warum so faul?}
    \begin{exercise}[a)]
    Programmieren Sie eine Funktion, die für einen Integer-Wert prüft, ob dieser eine Primzahl ist.
    \end{exercise}
    \begin{solve}[a)]
    \begin{plainhaskell}
is_prime :: Integer -> Bool
is_prime n
    | n <= 1 = False
    | n == 2 = True
    | n `mod` 2 == 0 = False
    | otherwise = null [ x | x <- [3,5 .. floor (sq!*\tikzmark{sqrt}*!rt (fromIntegral n))], n `mod` x == 0]
    \end{plainhaskell}
    \end{solve}
    \begin{tikzpicture}[overlay,remember picture]
        \node[above=0cm of pic cs:sqrt] (n0) {};
        \node[above=2em of n0,inner sep=2pt,draw,rounded corners,anchor=south] (n1) {\shortstack{$\forall n=x\cdot y:$ \\ $x\geq\sqrt{n}\implies y \leq\sqrt{n}$}};
        \draw (n0) edge[->,dashed] (n1.south);
    \end{tikzpicture}
\end{frame}
\resetframecounters
\begin{frame}[fragile]{Warum so faul?}
    \begin{exercise}[b)]
    Definieren Sie unter Nutzung von \bhaskell{is_prime} eine Liste aller Primzahlen.
    \end{exercise}
    \begin{solve}[b)]
    \begin{plainhaskell}
primes :: [Integer]
primes = [ n | n <- [2..], is_prime n ]
    \end{plainhaskell}
    oder effizienter:
    \begin{plainhaskell}
primes :: [Integer]
primes = [2] ++ [ n | n <- [3,5 ..], is_prime n ]
    \end{plainhaskell}
    \end{solve}
\end{frame}
\resetframecounters
\begin{frame}[fragile]{Warum so faul?}
    \begin{exercise}[c)]
    Extrahieren Sie die 42. Primzahl.
    \end{exercise}
    \begin{solve}[c)]
    \begin{plainhaskell}
prime_no :: Int -> Integer
prime_no n = last (take n primes)
    \end{plainhaskell}
    \begin{plainbash}
> prime_no 42
181
    \end{plainbash}
    \end{solve}
\end{frame}