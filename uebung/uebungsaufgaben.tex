\documentclass[aspectratio=169,usepdftitle=true,11pt,ngerman,t]{beamer}

\usepackage[T1]{fontenc}
\usepackage[utf8]{inputenc}
\usepackage{microtype}

\newcommand\lqfirstname{Name}
\newcommand\lqfullname{Name Last}
\newcommand\lqemail{name.last}

\usetheme[darkmode,nobib]{digital-minimal}

\usepackage{fancyqr}
\usepackage[addons]{color-palettes}
\UsePalette{Peach}
\usepackage[cpalette,fakeminted]{sopra-listings}

% languages fix
\let\originalsolLoadLanguage\solLoadLanguage
\def\solLoadLanguage#1{%
    \def\temp{#1}
    \def\tempjava{java}
    \def\temptext{text}
    \ifx\temp\tempjava
        \input{../global/fix_language_java.cfg}
    \else\ifx\temp\temptext
        \input{../global/language_text.cfg}
    \else
        \originalsolLoadLanguage{#1}
    \fi\fi
}

\solLoadLanguage{java}
\solLoadLanguage{haskell}

%%%%%%%%%%%%%%%%%%%%%%%%%%%%%%%%%%%%%%%

\subtitle{Tutorium 12 -- Übungsaufgaben}
\date{17. Januar 2025}
\title{Informatik I}
\author{\lqfullname}

\outro{\centering{\textbf{Viel Erfolg beim Üben!}}}

%%%%%%%%%%%%%%%%%%%%%%%%%%%%%%%%%%%%%%%

\begin{document}

\lstfs{11}

\section{Altklausur WS21/22 Aufgabe 2}

\begin{frame}[fragile]{Java}\onslide<+->
    \begin{plainjava}
public static double arctan(double x) {
!*\onslide<+->*!	double result = 0;
	for (int k = 0; !*\onslide<+->*!Math.abs(
            Math.pow(x,2*k+1) / (2*k + 1)
        ) >= 0.000001; !*\onslide<+->*!k++) {
!*\onslide<+->*!	    result += Math.pow(-1, k) * Math.pow(x,2*k+1) / (2*k + 1);
	}
!*\onslide<+->*!	return result;
}
    \end{plainjava}
\end{frame}

\begin{frame}[fragile]{Java by ChatGPT}
    \begin{plainjava}
public static double arctan(double x) {
    if (Math.abs(x) > 1) throw new IllegalArgumentException();

    double sum = 0.0;
    double term;
    int k = 0;

    do {
        term = Math.pow(-1, k) * Math.pow(x, 2 * k + 1) / (2 * k + 1);
        sum += term;
        k++;
    } while (Math.abs(term) >= 0.000001);

    return sum;
}
    \end{plainjava}
\end{frame}

\begin{frame}[fragile]{Haskell}\onslide<+->
    \begin{plainhaskell}
!*\onslide<+->*!arctan :: Double -> Double
arctan x = arctansum 0 x

!*\onslide<+->*!arctansum :: Int -> Double -> Double
arctansum k x = !*\bfseries\textcolor{orange}{if x \^{} (2*k + 1) / fromIntegral (2*k + 1) < 0.000001 then 0 else}*! arctansumstep k x + arctansum (k+1) x

!*\onslide<+->*!arctansumstep :: Int -> Double -> Double
arctansumstep k x = (-1) ^ k * x ^ (2*k + 1) / !*\bfseries\textcolor{orange}{fromIntegral}*! (2*k + 1)
    \end{plainhaskell}
\end{frame}

\begin{frame}[fragile]{Haskell by ChatGPT}
    \begin{plainhaskell}
arctan :: Double -> Double
arctan x
  | abs x > 1 = error "x muss im Bereich [-1, 1] liegen!"
  | otherwise = arctanSum x 0 0

arctanSum :: Double -> Int -> Double -> Double
arctanSum x k sum
  | abs term < 0.000001 = sum
  | otherwise = arctanSum x (k + 1) (sum + term)
  where
    term = arctanTerm x k

arctanTerm :: Double -> Int -> Double
arctanTerm x k = ((-1) ^ k) * (x ^ (2 * k + 1)) / fromIntegral (2 * k + 1)
    \end{plainhaskell}
\end{frame}

\end{document}