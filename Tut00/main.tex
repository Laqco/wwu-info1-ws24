\documentclass[aspectratio=169,usepdftitle=true,t]{beamer}

\newcommand\lqfirstname{Name}
\newcommand\lqfullname{Name Last}
\newcommand\lqemail{name.last}

\usepackage[T1]{fontenc}
\usepackage[utf8]{inputenc}
\usepackage{microtype}
\usepackage[english,main=ngerman]{babel}
\usepackage{tikz}
\usepackage{hyperref}
\usepackage{fancyqr}
\usepackage{biblatex}

\usetikzlibrary{positioning}
\usetikzlibrary{arrows.meta}
\usetikzlibrary{decorations.pathreplacing,decorations.pathmorphing}

\usetheme[theorems]{colorful-dream}

\renewcommand\printbibliography[1][1]{\begin{center}\Huge\textbf{Viel Erfolg im Studium!}\end{center}}

%%%%%%%%%%%%
% SETTINGS %
%%%%%%%%%%%%

\title{Informatik I}
\subtitle{Tutorium 0}
\institute{Universität Münster}

\date{11. Oktober 2024}

\author{\lqfullname}
\email{{\lqemail}@uni-muenster.de}

\outro{Münster, \today}

%%%%%%%%%%%%%%%%%%
% TEMPLATE HACKS %
%%%%%%%%%%%%%%%%%%

% add credit to flo + this took way to long why is latex sometimes so weird :c
\addtobeamertemplate{title page}{}{%
\scriptsize\color{gray}\begin{tikzpicture}[overlay, remember picture]
    \node[above left = 1em and 1em of current page.south east,align=center] (n0) {\LaTeX-Vorlage von\\\href{https://github.com/EagleoutIce}{Florian Sihler}};
    \node[above = 0cm of n0] (n1) {\fancyqr[size=1.5cm,tight]{https://eagleoutice.github.io/beamer-themes/\#colorful-dream}};
\end{tikzpicture}}

% prevent error
\titleimage{\begin{tikzpicture}
    \node (n0) {};
\end{tikzpicture}}

%%%%%%%
% IDK %
%%%%%%%

% just an example command
\newcommand\twosplit[3][t]{%
    \begin{columns}[#1]
    \begin{column}{0.475\linewidth}
        #2
    \end{column}\hfill
    \begin{column}{0.475\linewidth}
        #3
    \end{column}
    \end{columns}
}

%%%%%%%%%
% NOTES %
%%%%%%%%%



%%%%%%%%%%%%
% DOCUMENT %
%%%%%%%%%%%%

% \section - Titelfolie

% \subsection - Footer subsection

% \begin{frame}{Was ist ein Algorithmus?} - Folie
% option: c - center vertically

% \begin{definition}[Algorithmus] - Defintion

% \begin{block}{Robust} - Block
% alertblock
% exampleblock

% \twosplit - links rechts

\begin{document}

\section{Willkommen}
\subsection{Vorstellungsrunde}

\begin{frame}{\only<1|beamer>{Über mich}\only<2|beamer>{Und ihr?}\only<0|handout:1>{Vorstellungsrunde}}
    \only<2>{\begin{itemize}
        \item Welches Fach?
        \item Warum Informatik?
        \item Vorkenntnisse?
    \end{itemize}
    \vspace*{\fill}}
    \begin{block}{\lqfirstname}
    \begin{itemize}
        \item \href{mailto:{\lqemail}@uni-muenster.de}{{\lqemail}@uni-muenster.de}
    \end{itemize}
    \end{block}
\end{frame}

\section{Organisatorisches}
\subsection{Tutorium}

\begin{frame}{Tutorium}
    \begin{itemize}
        \item<1-> \texttt{I10}: Freitag 10:15-11:45 Uhr (ct), SRZ 104
        \item<1-> \texttt{I13}: Freitag 12:30-14:00 Uhr (cct), SRZ 105
    \end{itemize}
    \vspace{\fill}
    \begin{itemize}
        \item<2-> keine Anwesenheitspflicht, aber \textbf{sinnvoll!}
        \item<2-> beste Gelegenheit, um Fragen zu stellen
    \end{itemize}
\end{frame}

\subsection{Übungsgruppen}

\begin{frame}{Abgabegruppen}
    \begin{itemize}
        \item<1-> Gruppengröße $\leq$ \textbf{3}\begin{itemize}
            \item \textbf{Alle} müssen arbeiten!
        \end{itemize}
        \only<2|handout:0>{\item \textbf{NEIN}}
        \item<3-> Nur im \textbf{selben} Tutorium.
        \vspace{\fill}
        \item<4-> \alert{\textbf{Altzulassung?}}
    \end{itemize}
    \begin{tikzpicture}[overlay, remember picture]
        \tikzset{
            disable rounded corners for decorations/.style={
                /pgf/every decoration/.style={
                    /tikz/sharp corners
                },
            }
        }

        \node[above left = 2.1cm and 1.7cm of current page.center] (n0) {};
        \node[above right = 1.6cm and 0.15cm of current page.center] (n1) {};
        \node[above right = 1.5cm and 0.17cm of current page.center,rotate=-10,rounded corners=8pt,draw,decorate,disable rounded corners for decorations,decoration={
            snake,
            amplitude=1pt,
            segment length=5pt
        }] (n2) {\footnotesize\hspace{-2pt}Meldet euch bei mir, wenn das nicht klappt!};

        \draw[decorate,decoration={snake,amplitude=0.6mm,segment length=5mm},->] (n0) -- (n1);
    \end{tikzpicture}
\end{frame}

\subsection{Abgabe}

\begin{frame}{Learnweb}
    \begin{itemize}
        \item<1-> Wöchentliche Übungszettel
        \item<2-> Abgabe der Lösungen
        \item<3-> \textit{\color{lightgray}Verbesserung von mir}
        \item<4-> Bewertung im Learnweb mit Kommentar
        \only<5-|handout:0>{\begin{tikzpicture}[overlay, remember picture]
            \node[draw] (n0) at (current page.center) {$\rightarrow$ Demo};
        \end{tikzpicture}}
        \vspace{\fill}
        \item<6-> Klausurzulassung: \textbf{50\%} der Punkte
    \end{itemize}
\end{frame}

\begin{frame}{Format}
    \only<2->{\begin{itemize}
        \item<2-> Text: \textbf{\texttt{\alert{PDF}}}\only<4->{\begin{itemize}
            \item tastet euch gerne mal an \href{https://www.latex-project.org/}{\LaTeX} ran
        \end{itemize}}
        \item<2-> Code: \textbf{\texttt{\alert{ZIP}}}\begin{itemize}
            \item<3-> Code im Text: \textbf{lesbar!} (d.h. eingerückt, Fotos vermeiden etc.)
            \item<3-> In den meisten IDEs gibt es eine Druck-Funktion
            \only<3->{\begin{tikzpicture}[overlay,remember picture]
                \node (n0) at (current page.center) {Vorlesung: \href{https://www.eclipse.org/downloads/}{eclipse}};
                \node[above left = 0.6cm and 2.9cm of current page.center] (ide) {};
                \node[above left = -0.4cm and -0.3cm of n0] (_n0) {};
                \draw[arrows={-Triangle[line width=10pt]}] (ide) to[bend right] (_n0);
            \end{tikzpicture}}
        \end{itemize}
    \end{itemize}
    \vspace{\fill}}
    \begin{block}{Wichtig}
        Abgaben im falschen Format werden nicht verbessert.
    \end{block}
\end{frame}

\subsection{Tutorium}

\begin{frame}{Inhalt}
    \begin{itemize}
        \item<1-> {\color{lightgray}\emph{``Besprechung der (Lösungen der) Übungsaufgaben''}}
        \item<2-> aufgetretene Probleme und Fragen klären
        \only<2|handout:0>{\vspace{\fill}\begin{block}{Eure Lösungen?}
            Darf ich Abgaben als Beispiele anführen? Mit oder ohne Quelle?
        \end{block}}
        \item<3-> Übungsblätter vorbesprechen
        \item<4-> zusammen Beispiele bearbeiten
        \only<0|handout:1>{\vspace{\fill}\begin{block}{Eure Lösungen zeigen?}
            Ich werde manchmal (mit Vorwarnung!) euren Code auf meinen Folien zeigen. Falls ihr das nicht wollt, schreibt mir das direkt oder in einem Kommentar in eurem Code!
        \end{block}}
    \end{itemize}
\end{frame}

\section{Bevor es losgeht...}
\subsection{Hardwarecheck}

\begin{frame}{Programmieren auf Laptop? iPad? Papier?!}
    \vspace{-2em}\twosplit{\only<1->{\begin{block}{gut}
        Laptop, Macbook, PC
    \end{block}}}{\only<2->{\begin{block}{schlecht}
        Tablet, iPad, Handy
    \end{block}}}
    \vspace{\fill}
    \only<3->{\begin{block}{PC-Pool}
        An der Brücke gibt es einen PC-Pool, an dem ihr arbeiten könnt.
    \end{block}}
    \only<0|handout:1>{\begin{block}{ASTA Laptop}
        \begin{minipage}{0.85\linewidth}
            Der ASTA kann euch auch Laptops für zwei Monate verleihen.
        \end{minipage}\nolinebreak\hfill\begin{minipage}{0.107\linewidth}
            \vspace{0.107cm}\fancyqr[size=1.5cm]{https://www.asta.ms/laptop-verleih}
        \end{minipage}\end{block}}
    \vspace{\fill}
    \only<4->{Auf Papier werdet ihr nur in der Klausur programmieren müssen, also keine Sorge!}
\end{frame}

\subsection{Hilfe}

\begin{frame}{Hilfe ich komm' nicht weiter!}
    \only<1->{\begin{block}{\href{https://www.uni-muenster.de/Informatik/learningcenter/}{LearningCenter}}
        \begin{minipage}{0.85\linewidth}
            Im LearningCenter Informatik könnt ihr Hilfestellung zum\\Studienstart und im Studium bekommen. Die Öffnungszeiten findet ihr auf der Website.
        \end{minipage}\nolinebreak\hfill\begin{minipage}{0.107\linewidth}
            \vspace{0.107cm}\fancyqr[size=1.5cm]{https://www.uni-muenster.de/Informatik/learningcenter/}
        \end{minipage}
    \end{block}}
    \vspace{\fill}
    \only<2->{\begin{block}{Zusammen schafft man mehr!}
        Arbeitet mit euren Übungsgruppen zusammen, fragt im Learnweb nach und fragt vor allem mich!
    \end{block}}
\end{frame}

\subsection{Fragen}

\begin{frame}<presentation:0|handout:1>{Nachtrag}
    \begin{block}{KI mit Stift und Papier?}
        \begin{minipage}{0.85\linewidth}
            Ein kleines Spiel für einsame Abende... Oder ein 60 Jahre altes Experiment!
        \end{minipage}\nolinebreak\hfill\begin{minipage}{0.107\linewidth}
            \vspace{0.107cm}\fancyqr[size=1.5cm]{https://www.i-am.ai/de/build-your-own-ai.html}
        \end{minipage}
    \end{block}
    \begin{block}{Folien}
        Ich werde meine vollständigen Folien ohne Gewähr auf Richtigkeit im Learnweb hochladen. Feedback gerne jederzeit an mich! :)
    \end{block}
\end{frame}

\end{document}