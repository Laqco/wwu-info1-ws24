\begin{frame}[fragile]\onslide<+->% init animation
    \begin{exercise}[a)]
    Programmieren Sie Algorithmus aus Aufgabe 2 von Übungsblatt 3 als rekursive Funktion. Welche Basisfälle lassen sich ableiten?
    \end{exercise}
    \begin{solve}[a)]
    \begin{plainhaskell}
euclid :: Integer -> Integer -> Integer
euclid m n
  | m < 0 || n < 0   = error "Eingabe negativ!"
  | m == 0           = n                -- Basisfall 1
  | n == 0           = m                -- Basisfall 2
  | m > n            = euclid (m - n) n -- rekursive Aufrufe
  | otherwise        = euclid m (n - m)
    \end{plainhaskell}
    \end{solve}
\end{frame}
\resetframecounters
\begin{frame}<handout:0>[fragile,noframenumbering]\onslide<+->% init animation
    \begin{exercise}[a)]
    Programmieren Sie Algorithmus aus Aufgabe 2 von Übungsblatt 3 als rekursive Funktion. Welche Basisfälle lassen sich ableiten?
    \end{exercise}
    \begin{solve}[a) I10\_a6: Näher am Pseudocode geht nicht!]
    \begin{plainhaskell}
euclid :: Integer -> Integer -> Integer
euclid m n
  | m < 0 || n < 0       = error "Eingabe negativ!"
  | m == 0               = n                -- Basisfall 1
  | n /= 0 && m > n      = euclid (m - n) n
  | n /= 0 && otherwise  = euclid m (n - m)
  | otherwise            = m                -- Basisfall 2
    \end{plainhaskell}
    \end{solve}
\end{frame}
\resetframecounters
\begin{frame}[noframenumbering]\onslide<+->% init animation
    \begin{exercise}[b)]
    Wieso terminiert Ihr Algorithmus?
    \end{exercise}
    \begin{solve}[b)]
    Im Fehlerfall oder bei Erreichen eines Basisfalls endet der Algorithmus.\par\bigskip
    In den rekursiven Aufrufen wird einer der Parameter um eine Zahl ungleich $0$ reduziert. Dies kann nur endlich oft geschehen, bevor ein Basisfall mit Terminierung erreicht wird.
    \onslide<+-|handout:0>\par\bigskip
    \textit{\lqfirstname s Ergänzung: (inpiriert von I10\_a4)}\par
    Es wird immer um eine Zahl reduziert, die kleiner als der Minuend ist. Damit kann der Fehlerfall nur beim initialen Aufruf erreicht werden; ansonsten wird immer schließlich ein Basisfall erreicht!
    \end{solve}
\end{frame}