\begin{frame}[fragile]\onslide<+->% init animation
    \begin{exercise}[a)]
    Schreiben Sie eine Funktion für \T{NAND}.
    \end{exercise}
    \begin{solve}[a)]
    \begin{plainhaskell}
nand :: Bool -> Bool -> Bool
nand a b = not (a && b)
    \end{plainhaskell}
    \end{solve}
    \onslide<+-|handout:0>
    \T{NAND}-Gatter sind ein vollständiges Logiksystem:\par\smallskip
    \begin{tabular}{l|c|c}
        \T{NOT} x & x \T{NAND} x\\
        \hline
        x \T{AND} y & (x \T{NAND} y) \T{NAND} (x \T{NAND} y) & \textit{NOT (x NAND y)}\\
        \hline
        x \T{OR} y & (x \T{NAND} x) \T{NAND} (y \T{NAND} y) & \textit{De-Morgan}\\
        \hline
        & $\cdots$
    \end{tabular}\par\smallskip
    $\Rightarrow$ Standartbaustein für Digitaltechnik
\end{frame}
\resetframecounters
\begin{frame}[fragile,noframenumbering]\onslide<+->% init animation
    \begin{exercise}[b)]
    Schreiben Sie eine Funktion für \T{XOR}.
    \end{exercise}
    \begin{solve}[b)]
    \begin{plainhaskell}
xor :: Bool -> Bool -> Bool
xor a b = (a && not b) || (not a && b)
    \end{plainhaskell}
    \end{solve}
\end{frame}
\resetframecounters
\begin{frame}<handout:0>[fragile,noframenumbering]\onslide<+->% init animation
    \begin{exercise}[b)]
    Schreiben Sie eine Funktion für \T{XOR}.
    \end{exercise}
    \begin{solve}[b) Idee von I13\_a1]
    \begin{plainhaskell}
xor :: Bool -> Bool -> Bool
xor a b = (a || b) && (nand a b)
    \end{plainhaskell}
    \end{solve}
\end{frame}