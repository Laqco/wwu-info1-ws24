\documentclass[aspectratio=169,usepdftitle=true,11pt,ngerman,t]{beamer}

\newcommand\lqfirstname{Name}
\newcommand\lqfullname{Name Last}
\newcommand\lqemail{name.last}

\errorcontextlines999999

\usepackage[T1]{fontenc}
\usepackage[utf8]{inputenc}
\usepackage{microtype}

\usepackage{amsmath,amssymb}
\usepackage{mathtools}
\usepackage{array}
\usepackage{booktabs}
\usepackage{cancel}
\usepackage{ulem}

\usepackage[prefix=]{xcolor-material}
\usepackage{graphics}
\usepackage{tcolorbox}

\usepackage[main=ngerman]{babel}

\usepackage{tikz}
\usetikzlibrary{positioning}
\usetikzlibrary{tikzmark}
\usetikzlibrary{calc}
\usetikzlibrary{automata}

\usepackage{csquotes}
\usepackage{hyperref}
\usepackage{algpseudocode}

% flo
\usepackage{fancyqr}
\usepackage{color-palettes}
\usepackage[encoding,noloadlangs,cpalette,numinpar,fakeminted]{sopra-listings}
\usepackage{code-animation}

\usetheme[theorems]{colorful-dream}

% TODO: cleanup?
\title{Informatik I}
\institute{Universität Münster}

\author{\lqfullname}
\email{\lqemail @uni-muenster.de}

\addtobeamertemplate{title page}{}{%
\scriptsize\color{gray}\begin{tikzpicture}[overlay, remember picture]
    \node[above left = 1em and 1em of current page.south east,align=center] (n0) {\LaTeX-Vorlage von\\\href{https://github.com/EagleoutIce}{Florian Sihler}};
    \node[above = 0cm of n0] (n1) {\fancyqr[size=1.5cm,tight]{https://eagleoutice.github.io/beamer-themes/\#colorful-dream}};
\end{tikzpicture}}

% TODO: more (specific)?
% better fractions
\newcommand\betterfrac[2]{
    \frac{#1}{\vphantom{#2}#2}
}

% exercise points % TODO
\newcommand\expoints[1]{\null\hfill\texttt{\scriptsize\color{lightgray}(#1)}}
% \omit\expoints{1}\ignorespaces
% a1

% title image: box
% ([vshift], inverse scale, width, content)
\newcommand\titleimagebox[4][0pt]{%
\newlength\tibheight \setlength\tibheight{4cm}
\newlength\tibtY \setlength\tibtY{-6em}
\titleimage{\begin{minipage}[c][#2\tibheight]{#3}\vspace*{#2\tibtY}%
    \vspace*{#1}\begin{tcolorbox}[boxrule=0.1pt]%
    \centering#4% TODO: bug: not centered if too wide
    \end{tcolorbox}%
\end{minipage}}}

% outro image: text
% (text)
\newcommand\outroimagetext[1]{\outro{Münster, \today%
\begin{tikzpicture}[remember picture,overlay]
    \node[align=center] at (current page.center) {#1};
\end{tikzpicture}}
}

% reset counters
\newcommand\resetframecounters{\addtocounter{exercise}{-1}\addtocounter{solve}{-1}}
\newcommand\resetsolve{\addtocounter{solve}{-1}}

% listing style for file input listings
\lstdefinestyle{files}{numbersep=-8pt,xleftmargin=-8pt,frame=none,numbers=none,backgroundcolor={}}

% small footnote
\newcommand{\nicefootnote}[1]{\tikz[overlay,remember picture]{\node[anchor=south,text width=\dimexpr\textwidth+2em\relax,align=center,above=1em of current page.south] (nicefootnote) {\hrule\vskip 1pt\footnotesize#1};}}

% italic for fonts that don't support itshape
\newcommand{\itforce}[1]{\tikz[baseline]{\node[xslant=0.2,anchor=base,inner sep=0pt]{#1};}}

% console example calls % TODO
% \newcommand<>\excall[1]{%
% \only<handout:0>{%
% \begin{tikzpicture}[overlay]
%     \node {%
%     \begin{tcolorbox}[boxrule=0.1pt,hbox,boxsep=2pt,left=0pt,right=0pt,top=0pt,bottom=0pt]
%         \ibash[numbers=none,frame=none,backgroundcolor={}]{#1}
%     \end{tcolorbox}
%     };
% \end{tikzpicture}
% }}

% 
% AnimateCode Commands
% 
% X        - zeile (0: all)
% /X:      - kommentar
% .X       - reset
% |X:      - reset + kommentar
%
% +        - continue/walk
% -        - repeat
% 
% oX : {X} - kein marker + weitere zeilen
% :X : {X} - weitere zeilen
% *C       - frei: \Line{X}\Comment{}\Location{start|end|k}\Reset
%
% \StoreAnimationTo\NAnim\StoreHandoutTo\NHandout\StoreTo\N
%
% handout={8/2,\CodeAnimGet{NAnim}/3} - can be used pre definition
%

\solLoadLanguage{python}

%%%%%%%%%%%%%%%%%%%%%%%%%%%%%%%%%%%%%%%

\subtitle{Tutorium 1}
\date{18. Oktober 2024}

\titleimagebox[10cm]{1}{0cm}{}

\outroimagetext{}

%%%%%%%%%%%%%%%%%%%%%%%%%%%%%%%%%%%%%%%

\begin{document}

\section[Hier meine niedergeschriebenen Notizen aus dem Tutorium]{Tutorium}

\subsection{Nachtrag zu letzter Woche}

\begin{frame}{Alte Version}
    \begin{itemize}
        \item Die Folien aus letzter Woche sind aktualisiert und im Learnweb hochgeladen. Schaut sie euch gerne nochmal an. Ich darf übrigens alle Folien hochladen, also auch mit \textit{meiner} Code-Lösung.
        \item Die Vorstellung der Übungsaufgaben ist nicht verpflichtend, ich werde euch höchstens mal im Tutorium Fragen stellen ^^
        \item Ihr könnt eure Abgabe noch nachträglich bearbeiten, also gebt lieber eine vorläufige Version ab, als am Ende die Frist zu verpassen.
        \item Ihr könnt mir gerne auch Feedback zur Vorlesung geben, denn ich bin euer direktester Kontakt zum Dozenten.
        \item Fragen am besten per Mail weil Learnweb bekomm ich nicht immer.        
    \end{itemize}
\end{frame}

\subsection{Übungsblatt 1}

\begin{frame}{Alte Version}
    \begin{itemize}
        \item Bei Aufgabe 2 ist das Benutzen eines Tools erlaubt, so wie es in der Aufgabenstellung steht. In allen anderen Fällen gilt weiterhin, dass ihr die Aufgaben selber lösen müsst; bei Verdacht auf Betrug müsst ihr in einer kleinen mündlichen Prüfung bei mir den Übungsbleitern eure Lösung verteidigen.
        \item Formulierung der Lösungen: möglichst direkt, nicht zu kompliziert machen, Lösungsweg gerne mit angeben
        \item Overflow bei Addition: fällt weg, falls fixer Bit-Bereich
    \end{itemize}
\end{frame}

\subsection{Übungsaufgaben}

\begin{frame}{Binäre Addition}
    \textbf{Wann muss ich den Übertrag abschneiden und wann nicht?}
    \begin{alignat*}{5}
        &1111\ +&\    &1 =&\ 1000&0\\
        &1111\ +&\ 000&1 =&\  000&0
    \end{alignat*}
    Führende Nullen implizieren, dass wir auf einer festen Bit-Breite rechnen.
    \bigskip\bigskip
    \begin{alignat*}{1}
        120 &+ 200 \text{ -- auf einem Byte}\\
        120 &+ 200 \equiv 64 \mod{265}
    \end{alignat*}
    $\Rightarrow$ ,,Abschneiden'' ist $\mod{}$.
\end{frame}

\begin{frame}{IEEE 754}
    \href{https://web.archive.org/web/20160806053349/http://www.csee.umbc.edu/~tsimo1/CMSC455/IEEE-754-2008.pdf}{Standart (2008 Version)}
    \bigskip
    \textbf{Formel für die Umrechnung:}
    $$(a_1\dots a_n-1), (a_0=1), \text{bias: }(2^{(r-1)}-1)$$
    \medskip
    Die Formel für den Bias lässt sich zu einem kleinen ,,Trick'' umwandeln:
    $$\text{Bias } 0\overline{1}$$
    Also z.B. bei $8$-Bit Bias: $01111111$
\end{frame}

\begin{frame}{IEEE 754}
    $$10101100_{IEEE-8} = -0,8125_{10}$$
    \bigskip\bigskip
    \begin{alignat*}{1}
        3,25_{10} = &01001010_{IEEE-8}\\
        &01001101_{IEEE-8} = 3,625_{10}
    \end{alignat*}
\end{frame}

\begin{frame}[fragile]{IEEE 754}
    IEEE 754 kann aufgrund der begrenzten Mantisse eine Ungenauigkeit haben
    $$0,4_{10} = 00011001_{IEEE-8} = 0,390625_{10}$$
    \medskip
    Das kann man z.B. auch in Python sehen:
    \begin{plainpython}
!*>{}>{}>*! 0.1 + 0.2
!*\itforce{\color{sol@colors@lst@numbers}0.30000000000000004}*!
    \end{plainpython}
\end{frame}

\begin{frame}{IEEE 754 -- denormalisiert}
    \href{https://de.wikipedia.org/wiki/IEEE\_754\#Interpretation\_des\_Zahlenformats}{\bfseries Definition}: $\overline{e}=0, e=1-\text{bias}, a_0=0$
    \bigskip
    $$00000001_{IEEE-8} = 0.015625_{10}$$
\end{frame}

\end{document}