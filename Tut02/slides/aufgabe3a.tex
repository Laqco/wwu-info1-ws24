\begin{frame}{Wir zählen Zwei und Zwei zusammen}
    \begin{exercise}
        Interpretieren Sie die Binärzahlen $0110\tikzmark{e3whitespace0}~0100$ und $1100\tikzmark{e3whitespace1}~0110$ im Zweierkomplement.
    \end{exercise}
    \pause
    \begin{tikzpicture}[overlay,remember picture]
        \node [inner sep=0,right = 2pt of pic cs:e3whitespace0] (w0) {};
        \node [inner sep=0,right = 2pt of pic cs:e3whitespace1] (w1) {};
        \node [inner sep=2pt,below right = 0.5em and 1.1em of pic cs:e3whitespace0,draw,rounded corners] (n1) {\textbf{\color{red}Zwei} Zahlen!};
        \node [inner sep=0,below left = -0.2cm and 0.05cm of n1] (n2) {};
        \node [inner sep=0,below right = -0.2cm and 0.05cm of n1] (n3) {};
        \draw (w0) edge[->,bend right=20] (n2);
        \draw (w1) edge[->,bend left=20] (n3);
    \end{tikzpicture}
    \pause
    \begin{solve}[a)]
        \begin{itemize}
            \item allgemeine Formel: $({\only<4->{\color{blue}}b_{n-1}} b_{n-2}\cdots b_0)_{(z)} = {\only<4->{\color{blue}}b_{n-1}} \cdot \tikzmark{a3a1}\left(-2^{n-1}\right)+(b_{n-2}\cdots b_0)_{(2)}$
            \pause\begin{tikzpicture}[overlay,remember picture]
                \node[inner sep=0,above right = 0.9em and 1ex of pic cs:a3a1] (n0) {$\mathrlap{\overbrace{\hphantom{-2^{n-1}}}^{\tikzmark{a3a2}}}$};
                \node[inner sep=0,above = -1.4pt of pic cs:a3a2] (n1) {\textcolor{white}{\scriptsize $\textcolor{cyan}{b_{n-1}}=1$}};
                \node[inner sep=0,above = 0.2em of n1] (n2) {\textcolor{white}{\scriptsize nur wenn}};
            \end{tikzpicture}\pause
            \item $0110~0100_{(2)}=2^6+2^5+2^2=100$
            \pause
            \item $\begin{aligned}[t]
                1100~0110_{(2)}=-2^7+0100~0110_{(2)}&=-2^7+2^6+2^2+2^1\\
                &=-128+70=-58
            \end{aligned}$
        \end{itemize}
    \end{solve}
\end{frame}