\begin{frame}{Ist das (wirklich so) schwierig?}
    \begin{exercise}
        Stellen Sie den Text \enquote{Ist das schwierig?} im ASCII-Format als Dezimal-, Binär- und Hexadezimalzahlen dar.
    \end{exercise}
    \pause
    \begin{solve}[mit \only<2->{\href{https://www.python.org/}}{Python}~---~\textit{nach Inspiration von \texttt{I10\_a1}}]
        \begin{minipage}{0.5\linewidth}\raggedright
            \bpython{ord('a')}\pause~~~\textcolor{lightgray}{\footnotesize\faAngleLeft~$97$}\pause\\
            \bpython{ord('b')}~~~\textcolor{lightgray}{\footnotesize\faAngleLeft~$98$}\pause\\
            \bpython{ord('c')}~~~\textcolor{lightgray}{\footnotesize\faAngleLeft~$99$}
        \end{minipage}\pause\begin{minipage}{0.5\linewidth}\raggedright
            \bpython{ord('A')}\pause~~~\textcolor{lightgray}{\footnotesize\faAngleLeft~$65$}\pause\\
            \bpython{ord('B')}~~~\textcolor{lightgray}{\footnotesize\faAngleLeft~$66$}\pause\\
            \bpython{ord('C')}~~~\textcolor{lightgray}{\footnotesize\faAngleLeft~$67$}
        \end{minipage}
    \end{solve}
\end{frame}
% copy to animate code
\addtocounter{exercise}{-1}\addtocounter{solve}{-1}% reset counters
\begin{frame}[noframenumbering,fragile]{Ist das (wirklich so) schwierig?}% handout:23
    \begin{exercise}
        Stellen Sie den Text \enquote{Ist das schwierig?} im ASCII-Format als Dezimal-, Binär- und Hexadezimalzahlen dar.
    \end{exercise}
    \begin{solve}[mit \href{https://www.python.org/}{Python}~---~\textit{nach Inspiration von \texttt{I10\_a1}}]\lstfs{9}
        \AnimateCode{onslide={%
            *\Line{1}\Location{end},
            %letter
            *\Line{2}\Location{end},
            *\Line{1}\Others{2}\Location{i0}\Reset\Comment{\T{character='I'}},
            *\Line{3}\Comment{\T{"73"}}\Location{end},
            %letter
            *\Line{2}\Location{end},
            *\Line{1}\Others{2}\Location{s0}\Reset\Comment{\T{character='s'}},
            *\Line{3}\Comment{\T{"73 115"}}\Location{end},
            %letter
            *\Line{2}\Location{end},
            *\Line{1}\Others{2}\Location{t0}\Reset\Comment{\T{character='t'}},
            *\Line{3}\Comment{\T{"73 115 116"}}\Location{end},
            %letter
            *\Line{2}\Location{end},
            *\Line{1}\Others{2}\Location{sp0}\Reset\Comment{\T{character=' '}},
            *\Line{3}\Comment{\T{"73 115 116 32"}}\Location{end},
            %letter
            *\Line{2}\Location{end},
            *\Line{1}\Others{2}\Location{d0}\Reset\Comment{\T{character='d'}},
            *\Line{3}\Comment{\T{"73 115 116 32 100"}}\Location{end},
            %letter
            *\Line{2}\Location{end},
            *\Line{1}\Others{2}\Location{a0}\Reset\Comment{\T{character='a'}},
            *\Line{3}\Comment{\T{"73 115 116 32 100 97"}}\Location{end},
            %letter
            *\Line{2}\Location{end},
            *\Line{1}\Others{2}\Location{s1}\Reset\Comment{\T{\only<-22>{character='s'}}},
            *\Line{3}\Comment{\T{"73 115 116 32 100 97 115\only<23->{~...}"}}\Location{end}
        }}
        \begin{plainpython}
string = "!*\AnimLoc[2pt]{i0}*!I!*\AnimLoc[2pt]{s0}*!s!*\AnimLoc[2pt]{t0}*!t!*\AnimLoc[2pt]{sp0}*! !*\AnimLoc[2pt]{d0}*!d!*\AnimLoc[2pt]{a0}*!a!*\AnimLoc[2pt]{s1}*!s schwierig?"

for character in string:
    print(ord(character),end=" ")
        \end{plainpython}
        \endAnimateCode
        \onslide*<23>{Und so wei\tikzmark{a2usw}ter...\begin{tikzpicture}[overlay,remember picture]
            \node[inner sep=0,below = 0cm of pic cs:a2usw] (n0) {};
            \node[inner sep=2pt,below right = 1em and 1cm of pic cs:a2usw,draw,rounded corners] (n1) {\bpython{bin(ord(character))}, \bpython{hex(ord(character))}};

            \draw (n0) edge[->,bend right=30] (n1.west);
        \end{tikzpicture}}
        \onslide<24|handout:0>Aber geht es einfacher?
    \end{solve}
\end{frame}