\documentclass[aspectratio=169,usepdftitle=true,11pt,ngerman,t]{beamer}

\newcommand\lqfirstname{Name}
\newcommand\lqfullname{Name Last}
\newcommand\lqemail{name.last}

\errorcontextlines999999

\usepackage[T1]{fontenc}
\usepackage[utf8]{inputenc}
\usepackage{microtype}

\usepackage{amsmath,amssymb}
\usepackage{mathtools}
\usepackage{array}
\usepackage{booktabs}
\usepackage{cancel}
\usepackage{ulem}

\usepackage[prefix=]{xcolor-material}
\usepackage{graphics}
\usepackage{tcolorbox}

\usepackage[main=ngerman]{babel}

\usepackage{tikz}
\usetikzlibrary{positioning}
\usetikzlibrary{tikzmark}
\usetikzlibrary{calc}
\usetikzlibrary{automata}

\usepackage{csquotes}
\usepackage{hyperref}
\usepackage{algpseudocode}

% flo
\usepackage{fancyqr}
\usepackage{color-palettes}
\usepackage[encoding,noloadlangs,cpalette,numinpar,fakeminted]{sopra-listings}
\usepackage{code-animation}

\usetheme[theorems]{colorful-dream}

% TODO: cleanup?
\title{Informatik I}
\institute{Universität Münster}

\author{\lqfullname}
\email{\lqemail @uni-muenster.de}

\addtobeamertemplate{title page}{}{%
\scriptsize\color{gray}\begin{tikzpicture}[overlay, remember picture]
    \node[above left = 1em and 1em of current page.south east,align=center] (n0) {\LaTeX-Vorlage von\\\href{https://github.com/EagleoutIce}{Florian Sihler}};
    \node[above = 0cm of n0] (n1) {\fancyqr[size=1.5cm,tight]{https://eagleoutice.github.io/beamer-themes/\#colorful-dream}};
\end{tikzpicture}}

% TODO: more (specific)?
% better fractions
\newcommand\betterfrac[2]{
    \frac{#1}{\vphantom{#2}#2}
}

% exercise points % TODO
\newcommand\expoints[1]{\null\hfill\texttt{\scriptsize\color{lightgray}(#1)}}
% \omit\expoints{1}\ignorespaces
% a1

% title image: box
% ([vshift], inverse scale, width, content)
\newcommand\titleimagebox[4][0pt]{%
\newlength\tibheight \setlength\tibheight{4cm}
\newlength\tibtY \setlength\tibtY{-6em}
\titleimage{\begin{minipage}[c][#2\tibheight]{#3}\vspace*{#2\tibtY}%
    \vspace*{#1}\begin{tcolorbox}[boxrule=0.1pt]%
    \centering#4% TODO: bug: not centered if too wide
    \end{tcolorbox}%
\end{minipage}}}

% outro image: text
% (text)
\newcommand\outroimagetext[1]{\outro{Münster, \today%
\begin{tikzpicture}[remember picture,overlay]
    \node[align=center] at (current page.center) {#1};
\end{tikzpicture}}
}

% reset counters
\newcommand\resetframecounters{\addtocounter{exercise}{-1}\addtocounter{solve}{-1}}
\newcommand\resetsolve{\addtocounter{solve}{-1}}

% listing style for file input listings
\lstdefinestyle{files}{numbersep=-8pt,xleftmargin=-8pt,frame=none,numbers=none,backgroundcolor={}}

% small footnote
\newcommand{\nicefootnote}[1]{\tikz[overlay,remember picture]{\node[anchor=south,text width=\dimexpr\textwidth+2em\relax,align=center,above=1em of current page.south] (nicefootnote) {\hrule\vskip 1pt\footnotesize#1};}}

% italic for fonts that don't support itshape
\newcommand{\itforce}[1]{\tikz[baseline]{\node[xslant=0.2,anchor=base,inner sep=0pt]{#1};}}

% console example calls % TODO
% \newcommand<>\excall[1]{%
% \only<handout:0>{%
% \begin{tikzpicture}[overlay]
%     \node {%
%     \begin{tcolorbox}[boxrule=0.1pt,hbox,boxsep=2pt,left=0pt,right=0pt,top=0pt,bottom=0pt]
%         \ibash[numbers=none,frame=none,backgroundcolor={}]{#1}
%     \end{tcolorbox}
%     };
% \end{tikzpicture}
% }}

% 
% AnimateCode Commands
% 
% X        - zeile (0: all)
% /X:      - kommentar
% .X       - reset
% |X:      - reset + kommentar
%
% +        - continue/walk
% -        - repeat
% 
% oX : {X} - kein marker + weitere zeilen
% :X : {X} - weitere zeilen
% *C       - frei: \Line{X}\Comment{}\Location{start|end|k}\Reset
%
% \StoreAnimationTo\NAnim\StoreHandoutTo\NHandout\StoreTo\N
%
% handout={8/2,\CodeAnimGet{NAnim}/3} - can be used pre definition
%

\solLoadLanguage{python}

%%%%%%%%%%%%%%%%%%%%%%%%%%%%%%%%%%%%%%%

\subtitle{Tutorium 2}
\date{25. Oktober 2024}

\titleimagebox{1}{7cm}{%
{\color{btcd@color@btext!50!btcd@color@white}\vspace*{-1em}%
$$(-1)^v\left(\sum_{i=0}^{n-1}a_i2^{-i}\right)2^{\overline{e}-(2^{n-1}-1)}$$%
}%
Spiel und Spaß mit \texttt{IEEE 754}}

\outroimagetext{\large\textbf{53 63 68 F6 6E 65 73 20 57 6F 63 68 65 6E 65 6E 64 65 20 3A 29}}

%%%%%%%%%%%%%%%%%%%%%%%%%%%%%%%%%%%%%%%

\begin{document}

\section[Noch zu letztem Tutorium...]{Tutorium 1}

\subsection{Übungsaufgabe}
\begin{frame}<handout:11>{$0.4$ \faAngleRight~IEEE 754 \faAngleRight~\textit{???}}
    \begin{alignat*}{4}
        0.4_{(10)} = 0.0\overline{1100}_{(2)} &=&~0.\tikzmark{mark0}0\tikzmark{mark1}1\tikzmark{mark2}&100110011..._{(2)}\\
        \only<1|handout:0>{&&&\hphantom{100110011..._{(2)}\cdot 2^{-2}}} % to align
        \only<2|handout:0>{&=& 1.&100110011..._{(2)}\cdot 2^{-2}}
        \only<3->{&=& {\color{red}+}1.&10\mathclap{\underbrace{\hphantom{1001}}_\text{\color{blue}Mantisse}}0110011..._{(2)}\cdot 2^{\color{green}-2}}
    \end{alignat*}
    \only<2->{\begin{tikzpicture}[overlay,remember picture]
        \tikzstyle{smallarrow} = []
        \only<3->{\tikzset{smallarrow/.append style={draw=green}}}
        \node [inner sep=0,below = 0.05cm of pic cs:mark0] (n0) {};
        \node [inner sep=0,below = 0.05cm of pic cs:mark1] (n1) {};
        \node [inner sep=0,below = 0.05cm of pic cs:mark2] (n2) {};
        \draw (n0) edge[->,bend right,smallarrow] (n1);
        \draw (n1) edge[->,bend right,smallarrow] (n2);
    \end{tikzpicture}}
    \\\par
    \only<4->{8-Bit IEEE 754 Gleitkommazahl: \texttt{\textcolor{red}{S} \textcolor{green}{EE\tikzmark{mark3}E} \textcolor{blue}{MMMM}}
    \begin{tikzpicture}[overlay,remember picture]
        \node [inner sep=0,below = 0.05ex of pic cs:mark3] (n0) {};
        \node [inner sep=2pt,right = 3.7cm of pic cs:mark3,draw,rounded corners] (n1) {$\text{bias: }2^{\only<4|handout:0>{r}\only<5->{3}-1}-1\only<5->{=3}$};
        \node [inner sep=0,below left = -0.2cm and 0.05cm of n1] (n2) {};
        \draw (n0) edge[->,bend right=10] (n2);
    \end{tikzpicture}
    \\\par}
    \[\begin{array}{w{c}{0.3\linewidth} w{c}{0.3\linewidth} w{c}{0.3\linewidth}}
        \only<6->{\textcolor{red}{+} & \textcolor{green}{-2}+3=\textcolor{orange}{1} & 1.\textcolor{blue}{1001}}
        \only<7|handout:0>{\\\textcolor{red}{0} & \hphantom{-2+3=1}\llap{\textcolor{orange}{001}} & \hphantom{1.}\textcolor{blue}{1001}}
        \only<8->{\\\textcolor{red}{0} & \hphantom{-2+3=1}\llap{\textcolor{orange}{001}} & \underbrace{\hphantom{1.}\textcolor{blue}{1001}}_{~~2^0+2^{-1}+2^{-4}}}
    \end{array}\]
    \[
        \only<9|handout:0>{\textcolor{red}{+}2^{\color{green}-2}\cdot(2^0+\textcolor{blue}{2^{-1}}+\textcolor{blue}{2^{-4}})\hphantom{~=0.390625}}
        \only<10->{\textcolor{red}{+}2^{\color{green}-2}\cdot(2^0+\textcolor{blue}{2^{-1}}+\textcolor{blue}{2^{-4}})=0.390625}%
        \only<11->{\rlap{\hspace{0.5em}\tiny$\rightarrow$2.34\% Abweichung}}
    \]
    \end{frame}
% TODO: farben: orange/grün

\section[Kann ja nur noch besser werden...]{Übungsblatt 1}

\subsection{Aufgabe 1}
\begin{frame}
    \begin{exercise}[\textit{schrittweise} Klammerung]
        Setzen Sie Klammern, um zu zeigen, in welcher Reihenfolge die Teilausdrücke ausgewertet werden, und geben Sie das Ergebnis der Auswertung an.
    \end{exercise}
    \begin{solve}[a) \lstcolorlet{numbers}{paletteD}\color{lightgray}\bhaskell{3 * 4 - 2 ^ 2 ^ 4}]
    \begin{itemize}
        \item \textbf{Potenz}: Bindungsstärke $8$, rechts-assoziativ:\hfill\bhaskell{3 * 4 - (2 ^ (2 ^ 4))}
        \item \textbf{Multiplikation}: Bindungsstärke $7$:\hfill\bhaskell{(3 * 4) - (2 ^ (2 ^ 4))}
        \item \textbf{Subtraktion}: Bindungsstärke $6$:\hfill\bhaskell{(3 * 4) - (2 ^ (2 ^ 4))}\\
        \hfill$12 - 65536 = -65524$
    \end{itemize}
    \end{solve}
\end{frame}

\subsection{Aufgabe 2}
\begin{frame}[fragile]
    \begin{exercise}[a) + b)]
        Programmieren Sie in Haskell einen zweistelligen, links-assoziativen Infix-Operator \textasciitilde\textasciitilde~mit der Bindungsstärke 5, der den Mittelwert zweier Float-Zahlen berechnet.
    \end{exercise}
    \begin{solve}[a) + b)]
    \begin{plainhaskell}
-- Bindungsstärke 5
infixl 5 ~~

-- Operator im Klammern
(~~) :: Float -> Float -> Float

-- Mittelwert: Summe, geteilt durch 2
x ~~ y = (x + y)/2
    \end{plainhaskell}
    \end{solve}
\end{frame}
% copy to cleanup my code
\addtocounter{exercise}{-1}\addtocounter{solve}{-1}% reset counters
\begin{frame}
    \begin{exercise}[c)]
        Geben Sie an, wie der Ausdruck \bhaskell{3 + 4 ~~ 5 * 6} ausgewertet wird. Welche Änderungen würden sich bei den Bindungsstärken $6$ oder $7$ ergeben?
    \end{exercise}
    \begin{solve}[c)]
    \begin{itemize}
        \item Bindungsstärke $5$: \bhaskell{*} \faAngleRight~\bhaskell{+} \faAngleRight{}~\bhaskell{\~~}:\hfill\bhaskell{(3 + 4) ~~ (5 * 6)}$= 18.5$
        \item Bindungsstärke $6$: \bhaskell{*} \faAngleRight~\bhaskell{+}$\mid$\bhaskell{\~~}, also links-assoziativ:\\\hfill\bhaskell{(3 + 4) ~~ (5 * 6)}$=18.5$
        \item Bindungsstärke $7$: \bhaskell{*}$\mid$\bhaskell{\~~} \faAngleRight~\bhaskell{+}, also links-assoziativ:\\\hfill\bhaskell{3 + ((4 ~~ 5) * 6)}$=30$
    \end{itemize}
    \end{solve}
\end{frame}

\subsection{Aufgabe 2 - Binär}
\addtocounter{exercise}{-1}\addtocounter{solve}{-1} % reset counters
\begin{frame}{Vielleicht geht es ja auch einfacher...}
    \begin{exercise}[\enquote{Ist das schwierig?} \faAngleRight~ASCII (Dec, Bin, Hex)]\end{exercise}
    \begin{solve}[Binär c:]
        \only<1|handout:0>{Wir erinnern uns:% TODO: command for box like this
        \nolinebreak\null\hfill\raisebox{-2.5em}{\scalebox{0.7}{\begin{tcolorbox}
            \begin{minipage}{0.5\linewidth}\raggedright
                \bpython{ord('a')}~~~\textcolor{lightgray}{\footnotesize\faAngleLeft~$97$}\\
                \bpython{ord('b')}~~~\textcolor{lightgray}{\footnotesize\faAngleLeft~$98$}\\
                \bpython{ord('c')}~~~\textcolor{lightgray}{\footnotesize\faAngleLeft~$99$}
            \end{minipage}\begin{minipage}{0.5\linewidth}\raggedright
                \bpython{ord('A')}~~~\textcolor{lightgray}{\footnotesize\faAngleLeft~$65$}\\
                \bpython{ord('B')}~~~\textcolor{lightgray}{\footnotesize\faAngleLeft~$66$}\\
                \bpython{ord('C')}~~~\textcolor{lightgray}{\footnotesize\faAngleLeft~$67$}
            \end{minipage}
        \end{tcolorbox}}}}
        \only<2->{
        Schauen wir uns das mal genauer an:\pause
        \begin{itemize}
            \item<3-> $\mathtt{'a'}=97_{(10)}={\only<7->{\color{purple}}011}0~0001_{(2)}$
            \item<5-> $\mathtt{'b'}=98_{(10)}={\only<7->{\color{purple}}011}0~0010_{(2)}$
            \item<7-> $\cdots$
            \item<4-> $\mathtt{'A'}=65_{(10)}={\only<7->{\color{purple}}010}0~0001_{(2)}$
            \item<6-> $\mathtt{'B'}=66_{(10)}={\only<7->{\color{purple}}0\tikzmark{binmark0}10}0~0010_{(2)}$
            \item<7-> $\cdots$
        \end{itemize}}
    \end{solve}
    \only<7->{\begin{tikzpicture}[overlay,remember picture]
        \node [inner sep=0,below right = 0.05cm and 2pt of pic cs:binmark0] (n0) {};
        \node [inner sep=2pt,below right = 0em and 8em of pic cs:binmark0,draw=purple,rounded corners] (n1) {\begin{minipage}{0.5\textwidth}
            \centering\scriptsize Da unser Alphabet $26_{(10)}=11010_{(2)}$ Buchstaben\\hat, müssen wir $5$ Stellen rechts freihalten!
        \end{minipage}};
        \node [inner sep=0,below left = -0.35cm and 0.05cm of n1] (n2) {};
        \draw (n0) edge[->,bend right=20] (n2);
    \end{tikzpicture}}
\end{frame}
% newframe for better code
\addtocounter{exercise}{-1}\addtocounter{solve}{-1} % reset counters
\begin{frame}[noframenumbering]{Vielleicht geht es ja auch einfacher...}
    \begin{exercise}[\enquote{Ist das schwierig?} \faAngleRight~ASCII (Dec, Bin, Hex)]\end{exercise}
    \begin{solve}[Binär c:]
        Das heißt:\begin{itemize}
            \item<2-> Kleinbuchstaben: $\textcolor{purple}{011}0\only<4|handout:0>{\mathllap{\underbrace{\hphantom{0110}}_\text{\color{lightgray}0x\color{blue}6}}}~0001\only<4|handout:0>{\mathllap{\underbrace{\hphantom{0001}}_\text{\color{lightgray}0x\color{blue}1}}}$~---~$\textcolor{purple}{011}1\only<4|handout:0>{\mathllap{\underbrace{\hphantom{0111}}_\text{\color{lightgray}0x\color{blue}7}}}~1010\only<4|handout:0>{\mathllap{\underbrace{\hphantom{1010}}_\text{\color{lightgray}0x\color{blue}A}}}$\only<4|handout:0>{\vspace{-3pt}}\\
            \hphantom{Kleinbuchstaben: }\only<5->{$\color{lightgray}0x\color{blue}61$~---~$\color{lightgray}0x\color{blue}7A$}\\
            \hphantom{Kleinbuchstaben: }\only<6->{$97$~---~$122$}
            \item<3-> Großbuchstaben: $\textcolor{purple}{010}0\only<4|handout:0>{\mathllap{\underbrace{\hphantom{0100}}_\text{\color{lightgray}0x\color{blue}4}}}~0001\only<4|handout:0>{\mathllap{\underbrace{\hphantom{0001}}_\text{\color{lightgray}0x\color{blue}1}}}$~---~$\textcolor{purple}{010}1\only<4|handout:0>{\mathllap{\underbrace{\hphantom{0101}}_\text{\color{lightgray}0x\color{blue}5}}}~1010\only<4|handout:0>{\mathllap{\underbrace{\hphantom{1010}}_\text{\color{lightgray}0x\color{blue}A}}}$\only<4|handout:0>{\vspace{-3pt}}\\
            \hphantom{Großbuchstaben: }\only<5->{$\color{lightgray}0x\color{blue}41$~---~$\color{lightgray}0x\color{blue}5A$}\\
            \hphantom{Großbuchstaben: }\only<6->{$65$~---~$90$}
        \end{itemize}
    \end{solve}
\end{frame}

\subsection{Aufgabe 3.a}
\begin{frame}{Wir zählen Zwei und Zwei zusammen}
    \begin{exercise}
        Interpretieren Sie die Binärzahlen $0110\tikzmark{e3whitespace0}~0100$ und $1100\tikzmark{e3whitespace1}~0110$ im Zweierkomplement.
    \end{exercise}
    \pause
    \begin{tikzpicture}[overlay,remember picture]
        \node [inner sep=0,right = 2pt of pic cs:e3whitespace0] (w0) {};
        \node [inner sep=0,right = 2pt of pic cs:e3whitespace1] (w1) {};
        \node [inner sep=2pt,below right = 0.5em and 1.1em of pic cs:e3whitespace0,draw,rounded corners] (n1) {\textbf{\color{red}Zwei} Zahlen!};
        \node [inner sep=0,below left = -0.2cm and 0.05cm of n1] (n2) {};
        \node [inner sep=0,below right = -0.2cm and 0.05cm of n1] (n3) {};
        \draw (w0) edge[->,bend right=20] (n2);
        \draw (w1) edge[->,bend left=20] (n3);
    \end{tikzpicture}
    \pause
    \begin{solve}[a)]
        \begin{itemize}
            \item allgemeine Formel: $({\only<4->{\color{blue}}b_{n-1}} b_{n-2}\cdots b_0)_{(z)} = {\only<4->{\color{blue}}b_{n-1}} \cdot \tikzmark{a3a1}\left(-2^{n-1}\right)+(b_{n-2}\cdots b_0)_{(2)}$
            \pause\begin{tikzpicture}[overlay,remember picture]
                \node[inner sep=0,above right = 0.9em and 1ex of pic cs:a3a1] (n0) {$\mathrlap{\overbrace{\hphantom{-2^{n-1}}}^{\tikzmark{a3a2}}}$};
                \node[inner sep=0,above = -1.4pt of pic cs:a3a2] (n1) {\textcolor{white}{\scriptsize $\textcolor{cyan}{b_{n-1}}=1$}};
                \node[inner sep=0,above = 0.2em of n1] (n2) {\textcolor{white}{\scriptsize nur wenn}};
            \end{tikzpicture}\pause
            \item $0110~0100_{(2)}=2^6+2^5+2^2=100$
            \pause
            \item $\begin{aligned}[t]
                1100~0110_{(2)}=-2^7+0100~0110_{(2)}&=-2^7+2^6+2^2+2^1\\
                &=-128+70=-58
            \end{aligned}$
        \end{itemize}
    \end{solve}
\end{frame}

\subsection{Aufgabe 3.b}
\addtocounter{exercise}{-1}\addtocounter{solve}{-1} % reset counters
\begin{frame}{Wie in der Grundschule}
    \begin{exercise}
        Interpretieren Sie die Binärzahlen $0110\tikzmark{e3whitespace0}~0100$ und $1100\tikzmark{e3whitespace1}~0110$ im Zweierkomplement.
    \end{exercise}
    \begin{tikzpicture}[overlay,remember picture]
        \node [inner sep=0,right = 2pt of pic cs:e3whitespace0] (w0) {};
        \node [inner sep=0,right = 2pt of pic cs:e3whitespace1] (w1) {};
        \node [inner sep=2pt,below right = 0.5em and 1.1em of pic cs:e3whitespace0,draw,rounded corners] (n1) {\textbf{\color{red}Zwei} Zahlen!};
        \node [inner sep=0,below left = -0.2cm and 0.05cm of n1] (n2) {};
        \node [inner sep=0,below right = -0.2cm and 0.05cm of n1] (n3) {};
        \draw (w0) edge[->,bend right=20] (n2);
        \draw (w1) edge[->,bend left=20] (n3);
    \end{tikzpicture}
    \begin{solve}[b)]
        \vspace*{-2em}\begin{alignat*}{3}
             &\only<9|handout:0>{\textcolor{blue}}0\only<8|handout:0>{\textcolor{blue}}1\only<7|handout:0>{\textcolor{blue}}1\only<6|handout:0>{\textcolor{blue}}0~&\only<5|handout:0>{\textcolor{blue}}0\only<4|handout:0>{\textcolor{blue}}1\only<3|handout:0>{\textcolor{blue}}0\only<2|handout:0>{\textcolor{blue}}0\\
            +~&\tikzmark{saue1}\only<9|handout:0>{\textcolor{blue}}1\only<8|handout:0>{\textcolor{blue}}1\only<7|handout:0>{\textcolor{blue}}0\only<6|handout:0>{\textcolor{blue}}0~&\tikzmark{saue0}\only<5|handout:0>{\textcolor{blue}}0\only<4|handout:0>{\textcolor{blue}}1\only<3|handout:0>{\textcolor{blue}}1\only<2|handout:0>{\textcolor{blue}}0\\
            \addlinespace[-2\belowrulesep]\cmidrule[\lightrulewidth]{2-3}
             &\uncover<9->{\only<9|handout:0>{\textcolor{orange}}0}\uncover<8->{\only<8|handout:0>{\textcolor{orange}}0}\uncover<7->{\only<7|handout:0>{\textcolor{orange}}1}\uncover<6->{\only<6|handout:0>{\textcolor{orange}}0}~&\uncover<5->{\only<5|handout:0>{\textcolor{orange}}1}\uncover<4->{\only<4|handout:0>{\textcolor{orange}}0}\uncover<3->{\only<3|handout:0>{\textcolor{orange}}1}\uncover<2->{\only<2|handout:0>{\textcolor{orange}}0}
        \end{alignat*}
        \begin{tikzpicture}[overlay,remember picture]
            \node [inner sep=0,below right = 1em and 0.3ex of pic cs:saue0] (u0) {\only<4->{\tiny\only<5|handout:0>{\textcolor{blue}}{\only<4|handout:0>{\textcolor{orange}}1}}};
            \node [inner sep=0,below right = 1em and 0.3ex of pic cs:saue1] (u1) {\only<8->{\tiny\only<9|handout:0>{\textcolor{blue}}{\only<8|handout:0>{\textcolor{orange}}1}}};
            \node [inner sep=0,below left = 1em and 0.5ex of pic cs:saue1] (u2) {\only<9->{\tiny\only<9|handout:0>{\textcolor{orange}}1}};
        \end{tikzpicture}
        \onslide<10->$0010~1010_{(2)}=2^5+2^3+2^1=42=100-58$
    \end{solve}
\end{frame}

\subsection{Aufgabe 4}
\begin{frame}[fragile]{Jonglieren mit Variablen}
    \begin{exercise}
    Nutzen Sie in dieser Aufgabe die Kommandozeile zum Kompilieren und Ausführen von Java-Code.
    \medskip
    Lesen Sie zwei Zahlen von der Kommandozeile ein und speichern Sie diese als Variablen. Tauschen Sie die Werte der beiden Variablen und geben Sie den größeren Wert aus.
    \end{exercise}
\end{frame}
\begin{frame}[fragile]{Jonglieren mit Variablen}
    \begin{solve}[a) + b) + c)]
    \begin{plainjava}
// Deklaration (a)
int x, y;

// Eingabe (b)
x = IOTools.readInt("Eingabe x: ");
y = IOTools.readInt("Eingabe y: ");

// Ausgabe (c)
System.out.println("x: " + x + " - y: " + y);
    \end{plainjava}
    \end{solve}
\end{frame}
\resetframecounters
\begin{frame}[fragile]{Jonglieren mit Variablen}
    \begin{solve}[d)]
    \begin{plainjava}
// Tauschen (mit Hilfsvariable)
int z = x;
x = y;
y = z;

// Ausgabe
System.out.println("x: " + x + " - y: " + y);
    \end{plainjava}
    \end{solve}
\end{frame}
\resetframecounters
\begin{frame}[fragile]{Jonglieren mit Variablen}
    \begin{solve}[e)]
    \begin{plainjava}
// Vergleich
int largerValue;
if (x >= y)
    largerValue = x;
else
    largerValue = y;

System.out.println("Der größere Wert von beiden ist: " + largerValue);
    \end{plainjava}
\end{solve}
Ja, das ist eigentlich nur der ,,nicht-kleinere'' Wert!
\end{frame}
\resetframecounters
\begin{frame}[fragile]{Jonglieren mit Variablen}
    \vspace{-0.5\baselineskip}
    \begin{solve}[d) + e), aber schöner!]
    \begin{plainjava}
// Tauschen (mit XOR)
x = x ^ y;
y = x ^ y; // x ^ y ^ y = x
x = x ^ y; // x ^ y ^ x = y

// besserer Vergleich
int largerValue = x;
if (y > x) largerValue = y;

// oder direkt mit ternärem Operator
System.out.println("Der größere Wert von beiden ist: " + (x >= y) ? x : y);
    \end{plainjava}
    \end{solve}
\end{frame}

\section[Ein bisschen was zum Schluss]{Informationen}

\subsection{Termine}
\begin{frame}{nächste Woche}
    \begin{itemize}
        \item nächste Woche ist Feiertag
        \faAngleRight~Zoom Donnerstag 10:00 Uhr
        \item Meldet euch für eure Prüfungen an!
    \end{itemize}

    \vspace{\fill}\vspace{\fill} \Tiny Es ist \sout{04:30}\tikzmark{termin0} ich hab keine Lust mehr\tikzmark{termin1} :c
    \begin{tikzpicture}[overlay,remember picture]
        \node[above = 1ex of pic cs:termin0] (n0) {13:54 am Montag};
        \node[above = 1ex of pic cs:termin1] (n1) {hab ich immernoch nicht!};
    \end{tikzpicture}
\end{frame}

\subsection{Übungsblatt 2}
\begin{frame}{Die nächste Herausforderung...}\relax
    {\centering\huge Turingmaschinen\par}
    \vspace{1em}
    \begin{minipage}[t]{0.45\linewidth}\begin{tcolorbox}[boxrule=0.1pt]
        \hspace*{-1em}\begin{tikzpicture}[auto,remember picture]
            \node[state,initial,initial text=] (u2t00) {$q_0$};
            \node[state] (1) [right=of u2t00] {$q_1$};
            \path[->]
                (u2t00) edge[bend left] node{$x/x,S$} (1)
            ;
        \end{tikzpicture}\begin{tikzpicture}[overlay,remember picture]
            \node (x) [above left= 1em and 1.5em of u2t00] {\color{lightgray}$\forall x\in\Sigma$};
        \end{tikzpicture}
        \rule{\textwidth}{0.1pt}
        {\centering $\geq1$ Zeichen\par}
    \end{tcolorbox}\end{minipage}\hfill%
    \begin{minipage}[t]{0.45\linewidth}\begin{tcolorbox}[boxrule=0.1pt]
        \hspace*{-1em}\begin{tikzpicture}[auto,remember picture]
            \node[state,initial,initial text=] (u2t10) {$q_0$};
            \node[state] (1) [right=of u2t10] {$q_1$};
            \node[state] (2) [right=of 1] {$q_2$};
            \path[->]
                (u2t10) edge[bend left] node{$x/x,R$} (1)
                (1) edge[bend left] node{$x/x,L$} (2)
            ;
        \end{tikzpicture}\begin{tikzpicture}[overlay,remember picture]
            \node (x) [above left= 1em and -1em of u2t10] {\color{lightgray}$\forall x\in\Sigma$};
        \end{tikzpicture}
        \rule{\textwidth}{0.1pt}
        {\centering $\geq2$ Zeichen\par}
    \end{tcolorbox}\end{minipage}

    \begin{minipage}[t]{0.45\linewidth}%
        \vspace{0em}% WHY latex, whyyyy :(
        \begin{tcolorbox}[boxrule=0.1pt]
        \vspace*{-\baselineskip}\begin{alignat*}{2}
            \textcolor{lightgray}{\forall x\in\Sigma:~} &\delta(q_0,x)=(q_1,x,S)
        \end{alignat*}
    \end{tcolorbox}\end{minipage}\hfill%
    \begin{minipage}[t]{0.45\linewidth}%
        \vspace{0em}% WHY latex, whyyyy :(
        \begin{tcolorbox}[boxrule=0.1pt]
        \vspace*{-\baselineskip}\begin{alignat*}{2}
            \textcolor{lightgray}{\forall x\in\Sigma:~} &\delta(q_0,x)=(q_1,x,R)\\
            &\delta(q_1,x)=(q_2,x,L)
        \end{alignat*}
    \end{tcolorbox}\end{minipage}
\end{frame}

\end{document}